\documentclass[11pt]{article}

    \usepackage[breakable]{tcolorbox}
    \usepackage{parskip} % Stop auto-indenting (to mimic markdown behaviour)
    
    \usepackage{iftex}
    \ifPDFTeX
    	\usepackage[T1]{fontenc}
    	\usepackage{mathpazo}
    \else
    	\usepackage{fontspec}
    \fi

    % Basic figure setup, for now with no caption control since it's done
    % automatically by Pandoc (which extracts ![](path) syntax from Markdown).
    \usepackage{graphicx}
    % Maintain compatibility with old templates. Remove in nbconvert 6.0
    \let\Oldincludegraphics\includegraphics
    % Ensure that by default, figures have no caption (until we provide a
    % proper Figure object with a Caption API and a way to capture that
    % in the conversion process - todo).
    \usepackage{caption}
    \DeclareCaptionFormat{nocaption}{}
    \captionsetup{format=nocaption,aboveskip=0pt,belowskip=0pt}

    \usepackage{float}
    \floatplacement{figure}{H} % forces figures to be placed at the correct location
    \usepackage{xcolor} % Allow colors to be defined
    \usepackage{enumerate} % Needed for markdown enumerations to work
    \usepackage{geometry} % Used to adjust the document margins
    \usepackage{amsmath} % Equations
    \usepackage{amssymb} % Equations
    \usepackage{textcomp} % defines textquotesingle
    % Hack from http://tex.stackexchange.com/a/47451/13684:
    \AtBeginDocument{%
        \def\PYZsq{\textquotesingle}% Upright quotes in Pygmentized code
    }
    \usepackage{upquote} % Upright quotes for verbatim code
    \usepackage{eurosym} % defines \euro
    \usepackage[mathletters]{ucs} % Extended unicode (utf-8) support
    \usepackage{fancyvrb} % verbatim replacement that allows latex
    \usepackage{grffile} % extends the file name processing of package graphics 
                         % to support a larger range
    \makeatletter % fix for old versions of grffile with XeLaTeX
    \@ifpackagelater{grffile}{2019/11/01}
    {
      % Do nothing on new versions
    }
    {
      \def\Gread@@xetex#1{%
        \IfFileExists{"\Gin@base".bb}%
        {\Gread@eps{\Gin@base.bb}}%
        {\Gread@@xetex@aux#1}%
      }
    }
    \makeatother
    \usepackage[Export]{adjustbox} % Used to constrain images to a maximum size
    \adjustboxset{max size={0.9\linewidth}{0.9\paperheight}}

    % The hyperref package gives us a pdf with properly built
    % internal navigation ('pdf bookmarks' for the table of contents,
    % internal cross-reference links, web links for URLs, etc.)
    \usepackage{hyperref}
    % The default LaTeX title has an obnoxious amount of whitespace. By default,
    % titling removes some of it. It also provides customization options.
    \usepackage{titling}
    \usepackage{longtable} % longtable support required by pandoc >1.10
    \usepackage{booktabs}  % table support for pandoc > 1.12.2
    \usepackage[inline]{enumitem} % IRkernel/repr support (it uses the enumerate* environment)
    \usepackage[normalem]{ulem} % ulem is needed to support strikethroughs (\sout)
                                % normalem makes italics be italics, not underlines
    \usepackage{mathrsfs}
    

    
    % Colors for the hyperref package
    \definecolor{urlcolor}{rgb}{0,.145,.698}
    \definecolor{linkcolor}{rgb}{.71,0.21,0.01}
    \definecolor{citecolor}{rgb}{.12,.54,.11}

    % ANSI colors
    \definecolor{ansi-black}{HTML}{3E424D}
    \definecolor{ansi-black-intense}{HTML}{282C36}
    \definecolor{ansi-red}{HTML}{E75C58}
    \definecolor{ansi-red-intense}{HTML}{B22B31}
    \definecolor{ansi-green}{HTML}{00A250}
    \definecolor{ansi-green-intense}{HTML}{007427}
    \definecolor{ansi-yellow}{HTML}{DDB62B}
    \definecolor{ansi-yellow-intense}{HTML}{B27D12}
    \definecolor{ansi-blue}{HTML}{208FFB}
    \definecolor{ansi-blue-intense}{HTML}{0065CA}
    \definecolor{ansi-magenta}{HTML}{D160C4}
    \definecolor{ansi-magenta-intense}{HTML}{A03196}
    \definecolor{ansi-cyan}{HTML}{60C6C8}
    \definecolor{ansi-cyan-intense}{HTML}{258F8F}
    \definecolor{ansi-white}{HTML}{C5C1B4}
    \definecolor{ansi-white-intense}{HTML}{A1A6B2}
    \definecolor{ansi-default-inverse-fg}{HTML}{FFFFFF}
    \definecolor{ansi-default-inverse-bg}{HTML}{000000}

    % common color for the border for error outputs.
    \definecolor{outerrorbackground}{HTML}{FFDFDF}

    % commands and environments needed by pandoc snippets
    % extracted from the output of `pandoc -s`
    \providecommand{\tightlist}{%
      \setlength{\itemsep}{0pt}\setlength{\parskip}{0pt}}
    \DefineVerbatimEnvironment{Highlighting}{Verbatim}{commandchars=\\\{\}}
    % Add ',fontsize=\small' for more characters per line
    \newenvironment{Shaded}{}{}
    \newcommand{\KeywordTok}[1]{\textcolor[rgb]{0.00,0.44,0.13}{\textbf{{#1}}}}
    \newcommand{\DataTypeTok}[1]{\textcolor[rgb]{0.56,0.13,0.00}{{#1}}}
    \newcommand{\DecValTok}[1]{\textcolor[rgb]{0.25,0.63,0.44}{{#1}}}
    \newcommand{\BaseNTok}[1]{\textcolor[rgb]{0.25,0.63,0.44}{{#1}}}
    \newcommand{\FloatTok}[1]{\textcolor[rgb]{0.25,0.63,0.44}{{#1}}}
    \newcommand{\CharTok}[1]{\textcolor[rgb]{0.25,0.44,0.63}{{#1}}}
    \newcommand{\StringTok}[1]{\textcolor[rgb]{0.25,0.44,0.63}{{#1}}}
    \newcommand{\CommentTok}[1]{\textcolor[rgb]{0.38,0.63,0.69}{\textit{{#1}}}}
    \newcommand{\OtherTok}[1]{\textcolor[rgb]{0.00,0.44,0.13}{{#1}}}
    \newcommand{\AlertTok}[1]{\textcolor[rgb]{1.00,0.00,0.00}{\textbf{{#1}}}}
    \newcommand{\FunctionTok}[1]{\textcolor[rgb]{0.02,0.16,0.49}{{#1}}}
    \newcommand{\RegionMarkerTok}[1]{{#1}}
    \newcommand{\ErrorTok}[1]{\textcolor[rgb]{1.00,0.00,0.00}{\textbf{{#1}}}}
    \newcommand{\NormalTok}[1]{{#1}}
    
    % Additional commands for more recent versions of Pandoc
    \newcommand{\ConstantTok}[1]{\textcolor[rgb]{0.53,0.00,0.00}{{#1}}}
    \newcommand{\SpecialCharTok}[1]{\textcolor[rgb]{0.25,0.44,0.63}{{#1}}}
    \newcommand{\VerbatimStringTok}[1]{\textcolor[rgb]{0.25,0.44,0.63}{{#1}}}
    \newcommand{\SpecialStringTok}[1]{\textcolor[rgb]{0.73,0.40,0.53}{{#1}}}
    \newcommand{\ImportTok}[1]{{#1}}
    \newcommand{\DocumentationTok}[1]{\textcolor[rgb]{0.73,0.13,0.13}{\textit{{#1}}}}
    \newcommand{\AnnotationTok}[1]{\textcolor[rgb]{0.38,0.63,0.69}{\textbf{\textit{{#1}}}}}
    \newcommand{\CommentVarTok}[1]{\textcolor[rgb]{0.38,0.63,0.69}{\textbf{\textit{{#1}}}}}
    \newcommand{\VariableTok}[1]{\textcolor[rgb]{0.10,0.09,0.49}{{#1}}}
    \newcommand{\ControlFlowTok}[1]{\textcolor[rgb]{0.00,0.44,0.13}{\textbf{{#1}}}}
    \newcommand{\OperatorTok}[1]{\textcolor[rgb]{0.40,0.40,0.40}{{#1}}}
    \newcommand{\BuiltInTok}[1]{{#1}}
    \newcommand{\ExtensionTok}[1]{{#1}}
    \newcommand{\PreprocessorTok}[1]{\textcolor[rgb]{0.74,0.48,0.00}{{#1}}}
    \newcommand{\AttributeTok}[1]{\textcolor[rgb]{0.49,0.56,0.16}{{#1}}}
    \newcommand{\InformationTok}[1]{\textcolor[rgb]{0.38,0.63,0.69}{\textbf{\textit{{#1}}}}}
    \newcommand{\WarningTok}[1]{\textcolor[rgb]{0.38,0.63,0.69}{\textbf{\textit{{#1}}}}}
    
    
    % Define a nice break command that doesn't care if a line doesn't already
    % exist.
    \def\br{\hspace*{\fill} \\* }
    % Math Jax compatibility definitions
    \def\gt{>}
    \def\lt{<}
    \let\Oldtex\TeX
    \let\Oldlatex\LaTeX
    \renewcommand{\TeX}{\textrm{\Oldtex}}
    \renewcommand{\LaTeX}{\textrm{\Oldlatex}}
    % Document parameters
    % Document title
    \title{zadani}
    
    
    
    
    
% Pygments definitions
\makeatletter
\def\PY@reset{\let\PY@it=\relax \let\PY@bf=\relax%
    \let\PY@ul=\relax \let\PY@tc=\relax%
    \let\PY@bc=\relax \let\PY@ff=\relax}
\def\PY@tok#1{\csname PY@tok@#1\endcsname}
\def\PY@toks#1+{\ifx\relax#1\empty\else%
    \PY@tok{#1}\expandafter\PY@toks\fi}
\def\PY@do#1{\PY@bc{\PY@tc{\PY@ul{%
    \PY@it{\PY@bf{\PY@ff{#1}}}}}}}
\def\PY#1#2{\PY@reset\PY@toks#1+\relax+\PY@do{#2}}

\expandafter\def\csname PY@tok@w\endcsname{\def\PY@tc##1{\textcolor[rgb]{0.73,0.73,0.73}{##1}}}
\expandafter\def\csname PY@tok@c\endcsname{\let\PY@it=\textit\def\PY@tc##1{\textcolor[rgb]{0.25,0.50,0.50}{##1}}}
\expandafter\def\csname PY@tok@cp\endcsname{\def\PY@tc##1{\textcolor[rgb]{0.74,0.48,0.00}{##1}}}
\expandafter\def\csname PY@tok@k\endcsname{\let\PY@bf=\textbf\def\PY@tc##1{\textcolor[rgb]{0.00,0.50,0.00}{##1}}}
\expandafter\def\csname PY@tok@kp\endcsname{\def\PY@tc##1{\textcolor[rgb]{0.00,0.50,0.00}{##1}}}
\expandafter\def\csname PY@tok@kt\endcsname{\def\PY@tc##1{\textcolor[rgb]{0.69,0.00,0.25}{##1}}}
\expandafter\def\csname PY@tok@o\endcsname{\def\PY@tc##1{\textcolor[rgb]{0.40,0.40,0.40}{##1}}}
\expandafter\def\csname PY@tok@ow\endcsname{\let\PY@bf=\textbf\def\PY@tc##1{\textcolor[rgb]{0.67,0.13,1.00}{##1}}}
\expandafter\def\csname PY@tok@nb\endcsname{\def\PY@tc##1{\textcolor[rgb]{0.00,0.50,0.00}{##1}}}
\expandafter\def\csname PY@tok@nf\endcsname{\def\PY@tc##1{\textcolor[rgb]{0.00,0.00,1.00}{##1}}}
\expandafter\def\csname PY@tok@nc\endcsname{\let\PY@bf=\textbf\def\PY@tc##1{\textcolor[rgb]{0.00,0.00,1.00}{##1}}}
\expandafter\def\csname PY@tok@nn\endcsname{\let\PY@bf=\textbf\def\PY@tc##1{\textcolor[rgb]{0.00,0.00,1.00}{##1}}}
\expandafter\def\csname PY@tok@ne\endcsname{\let\PY@bf=\textbf\def\PY@tc##1{\textcolor[rgb]{0.82,0.25,0.23}{##1}}}
\expandafter\def\csname PY@tok@nv\endcsname{\def\PY@tc##1{\textcolor[rgb]{0.10,0.09,0.49}{##1}}}
\expandafter\def\csname PY@tok@no\endcsname{\def\PY@tc##1{\textcolor[rgb]{0.53,0.00,0.00}{##1}}}
\expandafter\def\csname PY@tok@nl\endcsname{\def\PY@tc##1{\textcolor[rgb]{0.63,0.63,0.00}{##1}}}
\expandafter\def\csname PY@tok@ni\endcsname{\let\PY@bf=\textbf\def\PY@tc##1{\textcolor[rgb]{0.60,0.60,0.60}{##1}}}
\expandafter\def\csname PY@tok@na\endcsname{\def\PY@tc##1{\textcolor[rgb]{0.49,0.56,0.16}{##1}}}
\expandafter\def\csname PY@tok@nt\endcsname{\let\PY@bf=\textbf\def\PY@tc##1{\textcolor[rgb]{0.00,0.50,0.00}{##1}}}
\expandafter\def\csname PY@tok@nd\endcsname{\def\PY@tc##1{\textcolor[rgb]{0.67,0.13,1.00}{##1}}}
\expandafter\def\csname PY@tok@s\endcsname{\def\PY@tc##1{\textcolor[rgb]{0.73,0.13,0.13}{##1}}}
\expandafter\def\csname PY@tok@sd\endcsname{\let\PY@it=\textit\def\PY@tc##1{\textcolor[rgb]{0.73,0.13,0.13}{##1}}}
\expandafter\def\csname PY@tok@si\endcsname{\let\PY@bf=\textbf\def\PY@tc##1{\textcolor[rgb]{0.73,0.40,0.53}{##1}}}
\expandafter\def\csname PY@tok@se\endcsname{\let\PY@bf=\textbf\def\PY@tc##1{\textcolor[rgb]{0.73,0.40,0.13}{##1}}}
\expandafter\def\csname PY@tok@sr\endcsname{\def\PY@tc##1{\textcolor[rgb]{0.73,0.40,0.53}{##1}}}
\expandafter\def\csname PY@tok@ss\endcsname{\def\PY@tc##1{\textcolor[rgb]{0.10,0.09,0.49}{##1}}}
\expandafter\def\csname PY@tok@sx\endcsname{\def\PY@tc##1{\textcolor[rgb]{0.00,0.50,0.00}{##1}}}
\expandafter\def\csname PY@tok@m\endcsname{\def\PY@tc##1{\textcolor[rgb]{0.40,0.40,0.40}{##1}}}
\expandafter\def\csname PY@tok@gh\endcsname{\let\PY@bf=\textbf\def\PY@tc##1{\textcolor[rgb]{0.00,0.00,0.50}{##1}}}
\expandafter\def\csname PY@tok@gu\endcsname{\let\PY@bf=\textbf\def\PY@tc##1{\textcolor[rgb]{0.50,0.00,0.50}{##1}}}
\expandafter\def\csname PY@tok@gd\endcsname{\def\PY@tc##1{\textcolor[rgb]{0.63,0.00,0.00}{##1}}}
\expandafter\def\csname PY@tok@gi\endcsname{\def\PY@tc##1{\textcolor[rgb]{0.00,0.63,0.00}{##1}}}
\expandafter\def\csname PY@tok@gr\endcsname{\def\PY@tc##1{\textcolor[rgb]{1.00,0.00,0.00}{##1}}}
\expandafter\def\csname PY@tok@ge\endcsname{\let\PY@it=\textit}
\expandafter\def\csname PY@tok@gs\endcsname{\let\PY@bf=\textbf}
\expandafter\def\csname PY@tok@gp\endcsname{\let\PY@bf=\textbf\def\PY@tc##1{\textcolor[rgb]{0.00,0.00,0.50}{##1}}}
\expandafter\def\csname PY@tok@go\endcsname{\def\PY@tc##1{\textcolor[rgb]{0.53,0.53,0.53}{##1}}}
\expandafter\def\csname PY@tok@gt\endcsname{\def\PY@tc##1{\textcolor[rgb]{0.00,0.27,0.87}{##1}}}
\expandafter\def\csname PY@tok@err\endcsname{\def\PY@bc##1{\setlength{\fboxsep}{0pt}\fcolorbox[rgb]{1.00,0.00,0.00}{1,1,1}{\strut ##1}}}
\expandafter\def\csname PY@tok@kc\endcsname{\let\PY@bf=\textbf\def\PY@tc##1{\textcolor[rgb]{0.00,0.50,0.00}{##1}}}
\expandafter\def\csname PY@tok@kd\endcsname{\let\PY@bf=\textbf\def\PY@tc##1{\textcolor[rgb]{0.00,0.50,0.00}{##1}}}
\expandafter\def\csname PY@tok@kn\endcsname{\let\PY@bf=\textbf\def\PY@tc##1{\textcolor[rgb]{0.00,0.50,0.00}{##1}}}
\expandafter\def\csname PY@tok@kr\endcsname{\let\PY@bf=\textbf\def\PY@tc##1{\textcolor[rgb]{0.00,0.50,0.00}{##1}}}
\expandafter\def\csname PY@tok@bp\endcsname{\def\PY@tc##1{\textcolor[rgb]{0.00,0.50,0.00}{##1}}}
\expandafter\def\csname PY@tok@fm\endcsname{\def\PY@tc##1{\textcolor[rgb]{0.00,0.00,1.00}{##1}}}
\expandafter\def\csname PY@tok@vc\endcsname{\def\PY@tc##1{\textcolor[rgb]{0.10,0.09,0.49}{##1}}}
\expandafter\def\csname PY@tok@vg\endcsname{\def\PY@tc##1{\textcolor[rgb]{0.10,0.09,0.49}{##1}}}
\expandafter\def\csname PY@tok@vi\endcsname{\def\PY@tc##1{\textcolor[rgb]{0.10,0.09,0.49}{##1}}}
\expandafter\def\csname PY@tok@vm\endcsname{\def\PY@tc##1{\textcolor[rgb]{0.10,0.09,0.49}{##1}}}
\expandafter\def\csname PY@tok@sa\endcsname{\def\PY@tc##1{\textcolor[rgb]{0.73,0.13,0.13}{##1}}}
\expandafter\def\csname PY@tok@sb\endcsname{\def\PY@tc##1{\textcolor[rgb]{0.73,0.13,0.13}{##1}}}
\expandafter\def\csname PY@tok@sc\endcsname{\def\PY@tc##1{\textcolor[rgb]{0.73,0.13,0.13}{##1}}}
\expandafter\def\csname PY@tok@dl\endcsname{\def\PY@tc##1{\textcolor[rgb]{0.73,0.13,0.13}{##1}}}
\expandafter\def\csname PY@tok@s2\endcsname{\def\PY@tc##1{\textcolor[rgb]{0.73,0.13,0.13}{##1}}}
\expandafter\def\csname PY@tok@sh\endcsname{\def\PY@tc##1{\textcolor[rgb]{0.73,0.13,0.13}{##1}}}
\expandafter\def\csname PY@tok@s1\endcsname{\def\PY@tc##1{\textcolor[rgb]{0.73,0.13,0.13}{##1}}}
\expandafter\def\csname PY@tok@mb\endcsname{\def\PY@tc##1{\textcolor[rgb]{0.40,0.40,0.40}{##1}}}
\expandafter\def\csname PY@tok@mf\endcsname{\def\PY@tc##1{\textcolor[rgb]{0.40,0.40,0.40}{##1}}}
\expandafter\def\csname PY@tok@mh\endcsname{\def\PY@tc##1{\textcolor[rgb]{0.40,0.40,0.40}{##1}}}
\expandafter\def\csname PY@tok@mi\endcsname{\def\PY@tc##1{\textcolor[rgb]{0.40,0.40,0.40}{##1}}}
\expandafter\def\csname PY@tok@il\endcsname{\def\PY@tc##1{\textcolor[rgb]{0.40,0.40,0.40}{##1}}}
\expandafter\def\csname PY@tok@mo\endcsname{\def\PY@tc##1{\textcolor[rgb]{0.40,0.40,0.40}{##1}}}
\expandafter\def\csname PY@tok@ch\endcsname{\let\PY@it=\textit\def\PY@tc##1{\textcolor[rgb]{0.25,0.50,0.50}{##1}}}
\expandafter\def\csname PY@tok@cm\endcsname{\let\PY@it=\textit\def\PY@tc##1{\textcolor[rgb]{0.25,0.50,0.50}{##1}}}
\expandafter\def\csname PY@tok@cpf\endcsname{\let\PY@it=\textit\def\PY@tc##1{\textcolor[rgb]{0.25,0.50,0.50}{##1}}}
\expandafter\def\csname PY@tok@c1\endcsname{\let\PY@it=\textit\def\PY@tc##1{\textcolor[rgb]{0.25,0.50,0.50}{##1}}}
\expandafter\def\csname PY@tok@cs\endcsname{\let\PY@it=\textit\def\PY@tc##1{\textcolor[rgb]{0.25,0.50,0.50}{##1}}}

\def\PYZbs{\char`\\}
\def\PYZus{\char`\_}
\def\PYZob{\char`\{}
\def\PYZcb{\char`\}}
\def\PYZca{\char`\^}
\def\PYZam{\char`\&}
\def\PYZlt{\char`\<}
\def\PYZgt{\char`\>}
\def\PYZsh{\char`\#}
\def\PYZpc{\char`\%}
\def\PYZdl{\char`\$}
\def\PYZhy{\char`\-}
\def\PYZsq{\char`\'}
\def\PYZdq{\char`\"}
\def\PYZti{\char`\~}
% for compatibility with earlier versions
\def\PYZat{@}
\def\PYZlb{[}
\def\PYZrb{]}
\makeatother


    % For linebreaks inside Verbatim environment from package fancyvrb. 
    \makeatletter
        \newbox\Wrappedcontinuationbox 
        \newbox\Wrappedvisiblespacebox 
        \newcommand*\Wrappedvisiblespace {\textcolor{red}{\textvisiblespace}} 
        \newcommand*\Wrappedcontinuationsymbol {\textcolor{red}{\llap{\tiny$\m@th\hookrightarrow$}}} 
        \newcommand*\Wrappedcontinuationindent {3ex } 
        \newcommand*\Wrappedafterbreak {\kern\Wrappedcontinuationindent\copy\Wrappedcontinuationbox} 
        % Take advantage of the already applied Pygments mark-up to insert 
        % potential linebreaks for TeX processing. 
        %        {, <, #, %, $, ' and ": go to next line. 
        %        _, }, ^, &, >, - and ~: stay at end of broken line. 
        % Use of \textquotesingle for straight quote. 
        \newcommand*\Wrappedbreaksatspecials {% 
            \def\PYGZus{\discretionary{\char`\_}{\Wrappedafterbreak}{\char`\_}}% 
            \def\PYGZob{\discretionary{}{\Wrappedafterbreak\char`\{}{\char`\{}}% 
            \def\PYGZcb{\discretionary{\char`\}}{\Wrappedafterbreak}{\char`\}}}% 
            \def\PYGZca{\discretionary{\char`\^}{\Wrappedafterbreak}{\char`\^}}% 
            \def\PYGZam{\discretionary{\char`\&}{\Wrappedafterbreak}{\char`\&}}% 
            \def\PYGZlt{\discretionary{}{\Wrappedafterbreak\char`\<}{\char`\<}}% 
            \def\PYGZgt{\discretionary{\char`\>}{\Wrappedafterbreak}{\char`\>}}% 
            \def\PYGZsh{\discretionary{}{\Wrappedafterbreak\char`\#}{\char`\#}}% 
            \def\PYGZpc{\discretionary{}{\Wrappedafterbreak\char`\%}{\char`\%}}% 
            \def\PYGZdl{\discretionary{}{\Wrappedafterbreak\char`\$}{\char`\$}}% 
            \def\PYGZhy{\discretionary{\char`\-}{\Wrappedafterbreak}{\char`\-}}% 
            \def\PYGZsq{\discretionary{}{\Wrappedafterbreak\textquotesingle}{\textquotesingle}}% 
            \def\PYGZdq{\discretionary{}{\Wrappedafterbreak\char`\"}{\char`\"}}% 
            \def\PYGZti{\discretionary{\char`\~}{\Wrappedafterbreak}{\char`\~}}% 
        } 
        % Some characters . , ; ? ! / are not pygmentized. 
        % This macro makes them "active" and they will insert potential linebreaks 
        \newcommand*\Wrappedbreaksatpunct {% 
            \lccode`\~`\.\lowercase{\def~}{\discretionary{\hbox{\char`\.}}{\Wrappedafterbreak}{\hbox{\char`\.}}}% 
            \lccode`\~`\,\lowercase{\def~}{\discretionary{\hbox{\char`\,}}{\Wrappedafterbreak}{\hbox{\char`\,}}}% 
            \lccode`\~`\;\lowercase{\def~}{\discretionary{\hbox{\char`\;}}{\Wrappedafterbreak}{\hbox{\char`\;}}}% 
            \lccode`\~`\:\lowercase{\def~}{\discretionary{\hbox{\char`\:}}{\Wrappedafterbreak}{\hbox{\char`\:}}}% 
            \lccode`\~`\?\lowercase{\def~}{\discretionary{\hbox{\char`\?}}{\Wrappedafterbreak}{\hbox{\char`\?}}}% 
            \lccode`\~`\!\lowercase{\def~}{\discretionary{\hbox{\char`\!}}{\Wrappedafterbreak}{\hbox{\char`\!}}}% 
            \lccode`\~`\/\lowercase{\def~}{\discretionary{\hbox{\char`\/}}{\Wrappedafterbreak}{\hbox{\char`\/}}}% 
            \catcode`\.\active
            \catcode`\,\active 
            \catcode`\;\active
            \catcode`\:\active
            \catcode`\?\active
            \catcode`\!\active
            \catcode`\/\active 
            \lccode`\~`\~ 	
        }
    \makeatother

    \let\OriginalVerbatim=\Verbatim
    \makeatletter
    \renewcommand{\Verbatim}[1][1]{%
        %\parskip\z@skip
        \sbox\Wrappedcontinuationbox {\Wrappedcontinuationsymbol}%
        \sbox\Wrappedvisiblespacebox {\FV@SetupFont\Wrappedvisiblespace}%
        \def\FancyVerbFormatLine ##1{\hsize\linewidth
            \vtop{\raggedright\hyphenpenalty\z@\exhyphenpenalty\z@
                \doublehyphendemerits\z@\finalhyphendemerits\z@
                \strut ##1\strut}%
        }%
        % If the linebreak is at a space, the latter will be displayed as visible
        % space at end of first line, and a continuation symbol starts next line.
        % Stretch/shrink are however usually zero for typewriter font.
        \def\FV@Space {%
            \nobreak\hskip\z@ plus\fontdimen3\font minus\fontdimen4\font
            \discretionary{\copy\Wrappedvisiblespacebox}{\Wrappedafterbreak}
            {\kern\fontdimen2\font}%
        }%
        
        % Allow breaks at special characters using \PYG... macros.
        \Wrappedbreaksatspecials
        % Breaks at punctuation characters . , ; ? ! and / need catcode=\active 	
        \OriginalVerbatim[#1,codes*=\Wrappedbreaksatpunct]%
    }
    \makeatother

    % Exact colors from NB
    \definecolor{incolor}{HTML}{303F9F}
    \definecolor{outcolor}{HTML}{D84315}
    \definecolor{cellborder}{HTML}{CFCFCF}
    \definecolor{cellbackground}{HTML}{F7F7F7}
    
    % prompt
    \makeatletter
    \newcommand{\boxspacing}{\kern\kvtcb@left@rule\kern\kvtcb@boxsep}
    \makeatother
    \newcommand{\prompt}[4]{
        {\ttfamily\llap{{\color{#2}[#3]:\hspace{3pt}#4}}\vspace{-\baselineskip}}
    }
    

    
    % Prevent overflowing lines due to hard-to-break entities
    \sloppy 
    % Setup hyperref package
    \hypersetup{
      breaklinks=true,  % so long urls are correctly broken across lines
      colorlinks=true,
      urlcolor=urlcolor,
      linkcolor=linkcolor,
      citecolor=citecolor,
      }
    % Slightly bigger margins than the latex defaults
    
    \geometry{verbose,tmargin=1in,bmargin=1in,lmargin=1in,rmargin=1in}
    
    

\begin{document}
    
    \maketitle
    
    

    
    Vítejte u druhého projektu do SUI. V rámci projektu Vás čeká několik
cvičení, v nichž budete doplňovat poměrně malé fragmenty kódu (místo je
vyznačeno pomocí \texttt{None} nebo \texttt{pass}). Pokud se v buňce s
kódem již něco nachází, využijte/neničte to. Buňky nerušte ani
nepřidávejte. Snažte se programovat hezky, ale jediná skutečně aktivně
zakázaná, vyhledávaná a -- i opakovaně -- postihovaná technika je
cyklení přes data (ať už explicitním cyklem nebo v rámci
\texttt{list}/\texttt{dict} comprehension), tomu se vyhýbejte jako čert
kříží a řešte to pomocí vhodných operací lineární algebry.

Až budete s řešením hotovi, vyexportujte ho (``Download as'') jako PDF i
pythonovský skript a ty odevzdejte pojmenované názvem týmu (tj. loginem
vedoucího). Dbejte, aby bylo v PDF všechno vidět (nezůstal kód za
okrajem stránky apod.).

U všech cvičení je uveden orientační počet řádků řešení. Berte ho prosím
opravdu jako orientační, pozornost mu věnujte, pouze pokud ho významně
překračujete.

    \begin{tcolorbox}[breakable, size=fbox, boxrule=1pt, pad at break*=1mm,colback=cellbackground, colframe=cellborder]
\prompt{In}{incolor}{1}{\boxspacing}
\begin{Verbatim}[commandchars=\\\{\}]
\PY{k+kn}{import} \PY{n+nn}{numpy} \PY{k}{as} \PY{n+nn}{np}
\PY{k+kn}{import} \PY{n+nn}{copy}
\PY{k+kn}{import} \PY{n+nn}{matplotlib}\PY{n+nn}{.}\PY{n+nn}{pyplot} \PY{k}{as} \PY{n+nn}{plt}
\PY{k+kn}{import} \PY{n+nn}{scipy}\PY{n+nn}{.}\PY{n+nn}{stats}
\end{Verbatim}
\end{tcolorbox}

    \section{Přípravné práce}\label{pux159uxedpravnuxe9-pruxe1ce}

Prvním úkolem v tomto projektu je načíst data, s nimiž budete pracovat.
Vybudujte jednoduchou třídu, která se umí zkonstruovat z cesty k
negativním a pozitivním příkladům, a bude poskytovat: - pozitivní a
negativní příklady (\texttt{dataset.pos}, \texttt{dataset.neg} o
rozměrech {[}N, 7{]}) - všechny příklady a odpovídající třídy
(\texttt{dataset.xs} o rozměru {[}N, 7{]}, \texttt{dataset.targets} o
rozměru {[}N{]})

K načítání dat doporučujeme využít \texttt{np.loadtxt()}. Netrapte se se
zapouzdřováním a gettery, berte třídu jako Plain Old Data.

Načtěte trénovací (\texttt{\{positives,negatives\}.trn}), validační
(\texttt{\{positives,negatives\}.val}) a testovací
(\texttt{\{positives,negatives\}.tst}) dataset, pojmenujte je po řadě
\texttt{train\_dataset}, \texttt{val\_dataset} a \texttt{test\_dataset}.

\textbf{(6 řádků)}

    \begin{tcolorbox}[breakable, size=fbox, boxrule=1pt, pad at break*=1mm,colback=cellbackground, colframe=cellborder]
\prompt{In}{incolor}{2}{\boxspacing}
\begin{Verbatim}[commandchars=\\\{\}]
\PY{k}{class} \PY{n+nc}{BinaryDataset}\PY{p}{:}
    \PY{k}{def} \PY{n+nf+fm}{\PYZus{}\PYZus{}init\PYZus{}\PYZus{}}\PY{p}{(}\PY{n+nb+bp}{self}\PY{p}{,} \PY{n}{d1}\PY{p}{,} \PY{n}{d2}\PY{p}{)}\PY{p}{:}
        \PY{n+nb+bp}{self}\PY{o}{.}\PY{n}{pos} \PY{o}{=} \PY{n}{np}\PY{o}{.}\PY{n}{loadtxt}\PY{p}{(}\PY{n}{d1}\PY{p}{)}
        \PY{n+nb+bp}{self}\PY{o}{.}\PY{n}{neg} \PY{o}{=} \PY{n}{np}\PY{o}{.}\PY{n}{loadtxt}\PY{p}{(}\PY{n}{d2}\PY{p}{)}
        \PY{n+nb+bp}{self}\PY{o}{.}\PY{n}{xs} \PY{o}{=} \PY{n}{np}\PY{o}{.}\PY{n}{append}\PY{p}{(}\PY{n+nb+bp}{self}\PY{o}{.}\PY{n}{pos}\PY{p}{,} \PY{n+nb+bp}{self}\PY{o}{.}\PY{n}{neg}\PY{p}{,} \PY{n}{axis}\PY{o}{=}\PY{l+m+mi}{0}\PY{p}{)}
        \PY{n+nb+bp}{self}\PY{o}{.}\PY{n}{targets} \PY{o}{=} \PY{n}{np}\PY{o}{.}\PY{n}{concatenate}\PY{p}{(}\PY{p}{(}\PY{n}{np}\PY{o}{.}\PY{n}{ones}\PY{p}{(}\PY{n+nb+bp}{self}\PY{o}{.}\PY{n}{pos}\PY{o}{.}\PY{n}{shape}\PY{p}{[}\PY{l+m+mi}{0}\PY{p}{]}\PY{p}{)}\PY{p}{,} \PY{n}{np}\PY{o}{.}\PY{n}{zeros}\PY{p}{(}\PY{n+nb+bp}{self}\PY{o}{.}\PY{n}{neg}\PY{o}{.}\PY{n}{shape}\PY{p}{[}\PY{l+m+mi}{0}\PY{p}{]}\PY{p}{)}\PY{p}{)}\PY{p}{)}

\PY{n}{train\PYZus{}dataset} \PY{o}{=} \PY{n}{BinaryDataset}\PY{p}{(}\PY{l+s+s1}{\PYZsq{}}\PY{l+s+s1}{positives.trn}\PY{l+s+s1}{\PYZsq{}}\PY{p}{,} \PY{l+s+s1}{\PYZsq{}}\PY{l+s+s1}{negatives.trn}\PY{l+s+s1}{\PYZsq{}}\PY{p}{)}
\PY{n}{val\PYZus{}dataset} \PY{o}{=} \PY{n}{BinaryDataset}\PY{p}{(}\PY{l+s+s1}{\PYZsq{}}\PY{l+s+s1}{positives.val}\PY{l+s+s1}{\PYZsq{}}\PY{p}{,} \PY{l+s+s1}{\PYZsq{}}\PY{l+s+s1}{negatives.val}\PY{l+s+s1}{\PYZsq{}}\PY{p}{)}
\PY{n}{test\PYZus{}dataset} \PY{o}{=} \PY{n}{BinaryDataset}\PY{p}{(}\PY{l+s+s1}{\PYZsq{}}\PY{l+s+s1}{positives.tst}\PY{l+s+s1}{\PYZsq{}}\PY{p}{,} \PY{l+s+s1}{\PYZsq{}}\PY{l+s+s1}{negatives.tst}\PY{l+s+s1}{\PYZsq{}}\PY{p}{)}

\PY{n+nb}{print}\PY{p}{(}\PY{l+s+s1}{\PYZsq{}}\PY{l+s+s1}{positives}\PY{l+s+s1}{\PYZsq{}}\PY{p}{,} \PY{n}{train\PYZus{}dataset}\PY{o}{.}\PY{n}{pos}\PY{o}{.}\PY{n}{shape}\PY{p}{)}
\PY{n+nb}{print}\PY{p}{(}\PY{l+s+s1}{\PYZsq{}}\PY{l+s+s1}{negatives}\PY{l+s+s1}{\PYZsq{}}\PY{p}{,} \PY{n}{train\PYZus{}dataset}\PY{o}{.}\PY{n}{neg}\PY{o}{.}\PY{n}{shape}\PY{p}{)}
\PY{n+nb}{print}\PY{p}{(}\PY{l+s+s1}{\PYZsq{}}\PY{l+s+s1}{xs}\PY{l+s+s1}{\PYZsq{}}\PY{p}{,} \PY{n}{train\PYZus{}dataset}\PY{o}{.}\PY{n}{xs}\PY{o}{.}\PY{n}{shape}\PY{p}{)}
\PY{n+nb}{print}\PY{p}{(}\PY{l+s+s1}{\PYZsq{}}\PY{l+s+s1}{targets}\PY{l+s+s1}{\PYZsq{}}\PY{p}{,} \PY{n}{train\PYZus{}dataset}\PY{o}{.}\PY{n}{targets}\PY{o}{.}\PY{n}{shape}\PY{p}{)}
\end{Verbatim}
\end{tcolorbox}

    \begin{Verbatim}[commandchars=\\\{\}]
positives (2280, 7)
negatives (6841, 7)
xs (9121, 7)
targets (9121,)
    \end{Verbatim}

    V řadě následujících cvičení budete pracovat s jedním konkrétním
příznakem. Naimplementujte proto funkci, která vykreslí histogram
rozložení pozitivních a negativních příkladů z jedné sady. Nezapomeňte
na legendu, ať je v grafu jasné, které jsou které. Funkci zavoláte
dvakrát, vykreslete histogram příznaku \texttt{5} -- tzn. šestého ze
sedmi -- pro trénovací a validační data

\textbf{(5 řádků)}

    \begin{tcolorbox}[breakable, size=fbox, boxrule=1pt, pad at break*=1mm,colback=cellbackground, colframe=cellborder]
\prompt{In}{incolor}{3}{\boxspacing}
\begin{Verbatim}[commandchars=\\\{\}]
\PY{n}{FOI} \PY{o}{=} \PY{l+m+mi}{5}  \PY{c+c1}{\PYZsh{} Feature Of Interest}

\PY{k}{def} \PY{n+nf}{plot\PYZus{}data}\PY{p}{(}\PY{n}{poss}\PY{p}{,} \PY{n}{negs}\PY{p}{)}\PY{p}{:}
    \PY{n}{plt}\PY{o}{.}\PY{n}{hist}\PY{p}{(}\PY{n}{poss}\PY{p}{,} \PY{n}{orientation}\PY{o}{=}\PY{l+s+s1}{\PYZsq{}}\PY{l+s+s1}{vertical}\PY{l+s+s1}{\PYZsq{}}\PY{p}{,} \PY{n}{bins}\PY{o}{=}\PY{l+m+mi}{200}\PY{p}{,} \PY{n}{color}\PY{o}{=}\PY{l+s+s1}{\PYZsq{}}\PY{l+s+s1}{green}\PY{l+s+s1}{\PYZsq{}}\PY{p}{,} \PY{n}{label}\PY{o}{=}\PY{l+s+s1}{\PYZsq{}}\PY{l+s+s1}{poss}\PY{l+s+s1}{\PYZsq{}}\PY{p}{,} \PY{n}{alpha}\PY{o}{=}\PY{l+m+mf}{0.5}\PY{p}{)}
    \PY{n}{plt}\PY{o}{.}\PY{n}{hist}\PY{p}{(}\PY{n}{negs}\PY{p}{,} \PY{n}{orientation}\PY{o}{=}\PY{l+s+s1}{\PYZsq{}}\PY{l+s+s1}{vertical}\PY{l+s+s1}{\PYZsq{}}\PY{p}{,} \PY{n}{bins}\PY{o}{=}\PY{l+m+mi}{200}\PY{p}{,} \PY{n}{color}\PY{o}{=}\PY{l+s+s1}{\PYZsq{}}\PY{l+s+s1}{red}\PY{l+s+s1}{\PYZsq{}}\PY{p}{,} \PY{n}{label}\PY{o}{=}\PY{l+s+s1}{\PYZsq{}}\PY{l+s+s1}{negs}\PY{l+s+s1}{\PYZsq{}}\PY{p}{,} \PY{n}{alpha}\PY{o}{=}\PY{l+m+mf}{0.5}\PY{p}{)}
    \PY{n}{plt}\PY{o}{.}\PY{n}{legend}\PY{p}{(}\PY{p}{)}
    \PY{n}{plt}\PY{o}{.}\PY{n}{show}\PY{p}{(}\PY{p}{)}

\PY{n}{plot\PYZus{}data}\PY{p}{(}\PY{n}{train\PYZus{}dataset}\PY{o}{.}\PY{n}{pos}\PY{p}{[}\PY{p}{:}\PY{p}{,} \PY{n}{FOI}\PY{p}{]}\PY{p}{,} \PY{n}{train\PYZus{}dataset}\PY{o}{.}\PY{n}{neg}\PY{p}{[}\PY{p}{:}\PY{p}{,} \PY{n}{FOI}\PY{p}{]}\PY{p}{)}
\PY{n}{plot\PYZus{}data}\PY{p}{(}\PY{n}{val\PYZus{}dataset}\PY{o}{.}\PY{n}{pos}\PY{p}{[}\PY{p}{:}\PY{p}{,} \PY{n}{FOI}\PY{p}{]}\PY{p}{,} \PY{n}{val\PYZus{}dataset}\PY{o}{.}\PY{n}{neg}\PY{p}{[}\PY{p}{:}\PY{p}{,} \PY{n}{FOI}\PY{p}{]}\PY{p}{)}
\end{Verbatim}
\end{tcolorbox}

    \begin{center}
    \adjustimage{max size={0.9\linewidth}{0.9\paperheight}}{zadani_files/zadani_5_0.png}
    \end{center}
    { \hspace*{\fill} \\}
    
    \begin{center}
    \adjustimage{max size={0.9\linewidth}{0.9\paperheight}}{zadani_files/zadani_5_1.png}
    \end{center}
    { \hspace*{\fill} \\}
    
    \subsubsection{Evaluace
klasifikátorů}\label{evaluace-klasifikuxe1torux16f}

Než přistoupíte k tvorbě jednotlivých klasifikátorů, vytvořte funkci pro
jejich vyhodnocování. Nechť se jmenuje \texttt{evaluate} a přijímá po
řadě klasifikátor, pole dat (o rozměrech {[}N, F{]}) a pole tříd
({[}N{]}). Jejím výstupem bude \emph{přesnost} (accuracy), tzn. podíl
správně klasifikovaných příkladů.

Předpokládejte, že klasifikátor poskytuje metodu
\texttt{.prob\_class\_1(data)}, která vrací pole posteriorních
pravděpodobností třídy 1 pro daná data. Evaluační funkce bude muset
provést tvrdé prahování (na hodnotě 0.5) těchto pravděpodobností a
srovnání získaných rozhodnutí s referenčními třídami. Využijte fakt, že
\texttt{numpy}ovská pole lze mj. porovnávat se skalárem.

\textbf{(3 řádky)}

    \begin{tcolorbox}[breakable, size=fbox, boxrule=1pt, pad at break*=1mm,colback=cellbackground, colframe=cellborder]
\prompt{In}{incolor}{4}{\boxspacing}
\begin{Verbatim}[commandchars=\\\{\}]
\PY{k}{def} \PY{n+nf}{evaluate}\PY{p}{(}\PY{n}{classifier}\PY{p}{,} \PY{n}{inputs}\PY{p}{,} \PY{n}{targets}\PY{p}{)}\PY{p}{:}
    \PY{n}{classified} \PY{o}{=} \PY{p}{(}\PY{n}{classifier}\PY{o}{.}\PY{n}{prob\PYZus{}class\PYZus{}1}\PY{p}{(}\PY{n}{inputs}\PY{p}{)} \PY{o}{\PYZgt{}} \PY{l+m+mf}{0.5}\PY{p}{)}
    \PY{n}{match\PYZus{}ratio} \PY{o}{=} \PY{n}{np}\PY{o}{.}\PY{n}{sum}\PY{p}{(}\PY{n}{classified}\PY{o}{.}\PY{n}{astype}\PY{p}{(}\PY{n+nb}{int}\PY{p}{)} \PY{o}{==} \PY{n}{targets}\PY{p}{)} \PY{o}{/} \PY{n}{targets}\PY{o}{.}\PY{n}{size}
    \PY{k}{return} \PY{n}{match\PYZus{}ratio}

\PY{k}{class} \PY{n+nc}{Dummy}\PY{p}{:}
    \PY{k}{def} \PY{n+nf}{prob\PYZus{}class\PYZus{}1}\PY{p}{(}\PY{n+nb+bp}{self}\PY{p}{,} \PY{n}{xs}\PY{p}{)}\PY{p}{:}
        \PY{k}{return} \PY{n}{np}\PY{o}{.}\PY{n}{asarray}\PY{p}{(}\PY{p}{[}\PY{l+m+mf}{0.2}\PY{p}{,} \PY{l+m+mf}{0.7}\PY{p}{,} \PY{l+m+mf}{0.7}\PY{p}{]}\PY{p}{)}

\PY{n+nb}{print}\PY{p}{(}\PY{n}{evaluate}\PY{p}{(}\PY{n}{Dummy}\PY{p}{(}\PY{p}{)}\PY{p}{,} \PY{k+kc}{None}\PY{p}{,} \PY{n}{np}\PY{o}{.}\PY{n}{asarray}\PY{p}{(}\PY{p}{[}\PY{l+m+mi}{0}\PY{p}{,} \PY{l+m+mi}{0}\PY{p}{,} \PY{l+m+mi}{1}\PY{p}{]}\PY{p}{)}\PY{p}{)}\PY{p}{)}  \PY{c+c1}{\PYZsh{} should be 0.66}
\end{Verbatim}
\end{tcolorbox}

    \begin{Verbatim}[commandchars=\\\{\}]
0.6666666666666666
    \end{Verbatim}

    \subsubsection{Baseline}\label{baseline}

Vytvořte klasifikátor, který ignoruje vstupní data. Jenom v konstruktoru
dostane třídu, kterou má dávat jako tip pro libovolný vstup.
Nezapomeňte, že jeho metoda \texttt{.prob\_class\_1(data)} musí vracet
pole správné velikosti.

\textbf{(4 řádky)}

    \begin{tcolorbox}[breakable, size=fbox, boxrule=1pt, pad at break*=1mm,colback=cellbackground, colframe=cellborder]
\prompt{In}{incolor}{5}{\boxspacing}
\begin{Verbatim}[commandchars=\\\{\}]
\PY{k}{class} \PY{n+nc}{PriorClassifier}\PY{p}{:}
    \PY{k}{def} \PY{n+nf+fm}{\PYZus{}\PYZus{}init\PYZus{}\PYZus{}}\PY{p}{(}\PY{n+nb+bp}{self}\PY{p}{,} \PY{n}{class\PYZus{}tip}\PY{p}{)}\PY{p}{:}
        \PY{n+nb+bp}{self}\PY{o}{.}\PY{n}{class\PYZus{}tip} \PY{o}{=} \PY{n}{class\PYZus{}tip}
    
    \PY{k}{def} \PY{n+nf}{prob\PYZus{}class\PYZus{}1}\PY{p}{(}\PY{n+nb+bp}{self}\PY{p}{,} \PY{n}{data}\PY{p}{)}\PY{p}{:}
        \PY{k}{return} \PY{n}{np}\PY{o}{.}\PY{n}{full}\PY{p}{(}\PY{n}{shape}\PY{o}{=}\PY{n}{val\PYZus{}dataset}\PY{o}{.}\PY{n}{targets}\PY{o}{.}\PY{n}{shape}\PY{p}{,} \PY{n}{fill\PYZus{}value}\PY{o}{=}\PY{n+nb+bp}{self}\PY{o}{.}\PY{n}{class\PYZus{}tip}\PY{p}{)}       

\PY{n}{baseline} \PY{o}{=} \PY{n}{PriorClassifier}\PY{p}{(}\PY{l+m+mi}{0}\PY{p}{)}
\PY{n}{val\PYZus{}acc} \PY{o}{=} \PY{n}{evaluate}\PY{p}{(}\PY{n}{baseline}\PY{p}{,} \PY{n}{val\PYZus{}dataset}\PY{o}{.}\PY{n}{xs}\PY{p}{[}\PY{p}{:}\PY{p}{,} \PY{n}{FOI}\PY{p}{]}\PY{p}{,} \PY{n}{val\PYZus{}dataset}\PY{o}{.}\PY{n}{targets}\PY{p}{)}
\PY{n+nb}{print}\PY{p}{(}\PY{l+s+s1}{\PYZsq{}}\PY{l+s+s1}{Baseline val acc:}\PY{l+s+s1}{\PYZsq{}}\PY{p}{,} \PY{n}{val\PYZus{}acc}\PY{p}{)}
\end{Verbatim}
\end{tcolorbox}

    \begin{Verbatim}[commandchars=\\\{\}]
Baseline val acc: 0.75
    \end{Verbatim}

    \section{Generativní
klasifikátory}\label{generativnuxed-klasifikuxe1tory}

V této části vytvoříte dva generativní klasifikátory, oba založené na
Gaussovu rozložení pravděpodobnosti.

Začněte implementací funce, která pro daná 1-D data vrátí Maximum
Likelihood odhad střední hodnoty a směrodatné odchylky Gaussova
rozložení, které data modeluje. Funkci využijte pro natrénovaní dvou
modelů: pozitivních a negativních příkladů. Získané parametry -- tzn.
střední hodnoty a směrodatné odchylky -- vypíšete.

\textbf{(1 řádek)}

    \begin{tcolorbox}[breakable, size=fbox, boxrule=1pt, pad at break*=1mm,colback=cellbackground, colframe=cellborder]
\prompt{In}{incolor}{6}{\boxspacing}
\begin{Verbatim}[commandchars=\\\{\}]
\PY{k}{def} \PY{n+nf}{mle\PYZus{}gauss\PYZus{}1d}\PY{p}{(}\PY{n}{data}\PY{p}{)}\PY{p}{:}
    \PY{k}{return} \PY{n}{np}\PY{o}{.}\PY{n}{mean}\PY{p}{(}\PY{n}{data}\PY{p}{)}\PY{p}{,} \PY{n}{np}\PY{o}{.}\PY{n}{std}\PY{p}{(}\PY{n}{data}\PY{p}{,} \PY{n}{ddof}\PY{o}{=}\PY{l+m+mi}{1}\PY{p}{)}

\PY{n}{mu\PYZus{}pos}\PY{p}{,} \PY{n}{std\PYZus{}pos} \PY{o}{=} \PY{n}{mle\PYZus{}gauss\PYZus{}1d}\PY{p}{(}\PY{n}{train\PYZus{}dataset}\PY{o}{.}\PY{n}{pos}\PY{p}{[}\PY{p}{:}\PY{p}{,} \PY{n}{FOI}\PY{p}{]}\PY{p}{)}
\PY{n}{mu\PYZus{}neg}\PY{p}{,} \PY{n}{std\PYZus{}neg} \PY{o}{=} \PY{n}{mle\PYZus{}gauss\PYZus{}1d}\PY{p}{(}\PY{n}{train\PYZus{}dataset}\PY{o}{.}\PY{n}{neg}\PY{p}{[}\PY{p}{:}\PY{p}{,} \PY{n}{FOI}\PY{p}{]}\PY{p}{)}

\PY{n+nb}{print}\PY{p}{(}\PY{l+s+s1}{\PYZsq{}}\PY{l+s+s1}{Pos mean: }\PY{l+s+si}{\PYZob{}:.2f\PYZcb{}}\PY{l+s+s1}{ std: }\PY{l+s+si}{\PYZob{}:.2f\PYZcb{}}\PY{l+s+s1}{\PYZsq{}}\PY{o}{.}\PY{n}{format}\PY{p}{(}\PY{n}{mu\PYZus{}pos}\PY{p}{,} \PY{n}{std\PYZus{}pos}\PY{p}{)}\PY{p}{)}
\PY{n+nb}{print}\PY{p}{(}\PY{l+s+s1}{\PYZsq{}}\PY{l+s+s1}{Neg mean: }\PY{l+s+si}{\PYZob{}:.2f\PYZcb{}}\PY{l+s+s1}{ std: }\PY{l+s+si}{\PYZob{}:.2f\PYZcb{}}\PY{l+s+s1}{\PYZsq{}}\PY{o}{.}\PY{n}{format}\PY{p}{(}\PY{n}{mu\PYZus{}neg}\PY{p}{,} \PY{n}{std\PYZus{}neg}\PY{p}{)}\PY{p}{)}
\end{Verbatim}
\end{tcolorbox}

    \begin{Verbatim}[commandchars=\\\{\}]
Pos mean: 0.48 std: 0.13
Neg mean: 0.17 std: 0.18
    \end{Verbatim}

    Ze získaných parametrů vytvořte \texttt{scipy}ovská gaussovská rozložení
\texttt{scipy.stats.norm}. S využitím jejich metody \texttt{.pdf()}
vytvořte graf, v němž srovnáte skutečné a modelové rozložení pozitivních
a negativních příkladů. Rozsah x-ové osy volte od -0.5 do 1.5 (využijte
\texttt{np.linspace}) a u volání \texttt{plt.hist()} nezapomeňte
nastavit \texttt{density=True}, aby byl histogram normalizovaný a dal se
srovnávat s modelem.

\textbf{(2 + 8 řádků)}

    \begin{tcolorbox}[breakable, size=fbox, boxrule=1pt, pad at break*=1mm,colback=cellbackground, colframe=cellborder]
\prompt{In}{incolor}{7}{\boxspacing}
\begin{Verbatim}[commandchars=\\\{\}]
\PY{n}{pos\PYZus{}dist} \PY{o}{=} \PY{n}{scipy}\PY{o}{.}\PY{n}{stats}\PY{o}{.}\PY{n}{norm}\PY{p}{(}\PY{n}{loc}\PY{o}{=}\PY{n}{mu\PYZus{}pos}\PY{p}{,} \PY{n}{scale}\PY{o}{=}\PY{n}{std\PYZus{}pos}\PY{p}{)}
\PY{n}{neg\PYZus{}dist} \PY{o}{=} \PY{n}{scipy}\PY{o}{.}\PY{n}{stats}\PY{o}{.}\PY{n}{norm}\PY{p}{(}\PY{n}{loc}\PY{o}{=}\PY{n}{mu\PYZus{}neg}\PY{p}{,} \PY{n}{scale}\PY{o}{=}\PY{n}{std\PYZus{}neg}\PY{p}{)}

\PY{n}{pos\PYZus{}samples} \PY{o}{=} \PY{n}{train\PYZus{}dataset}\PY{o}{.}\PY{n}{pos}\PY{p}{[}\PY{p}{:}\PY{p}{,} \PY{n}{FOI}\PY{p}{]}
\PY{n}{neg\PYZus{}samples} \PY{o}{=} \PY{n}{train\PYZus{}dataset}\PY{o}{.}\PY{n}{neg}\PY{p}{[}\PY{p}{:}\PY{p}{,} \PY{n}{FOI}\PY{p}{]}
\PY{n}{plt}\PY{o}{.}\PY{n}{hist}\PY{p}{(}\PY{n}{pos\PYZus{}samples}\PY{p}{,} \PY{n}{density}\PY{o}{=}\PY{k+kc}{True}\PY{p}{,} \PY{n}{bins}\PY{o}{=}\PY{l+m+mi}{20}\PY{p}{,} \PY{n}{color}\PY{o}{=}\PY{l+s+s1}{\PYZsq{}}\PY{l+s+s1}{deepskyblue}\PY{l+s+s1}{\PYZsq{}}\PY{p}{,} \PY{n}{label}\PY{o}{=}\PY{l+s+s1}{\PYZsq{}}\PY{l+s+s1}{pos}\PY{l+s+s1}{\PYZsq{}}\PY{p}{,} \PY{n}{alpha}\PY{o}{=}\PY{l+m+mf}{0.5}\PY{p}{)}
\PY{n}{plt}\PY{o}{.}\PY{n}{hist}\PY{p}{(}\PY{n}{neg\PYZus{}samples}\PY{p}{,} \PY{n}{density}\PY{o}{=}\PY{k+kc}{True}\PY{p}{,} \PY{n}{bins}\PY{o}{=}\PY{l+m+mi}{20}\PY{p}{,} \PY{n}{color}\PY{o}{=}\PY{l+s+s1}{\PYZsq{}}\PY{l+s+s1}{magenta}\PY{l+s+s1}{\PYZsq{}}\PY{p}{,} \PY{n}{label}\PY{o}{=}\PY{l+s+s1}{\PYZsq{}}\PY{l+s+s1}{neg}\PY{l+s+s1}{\PYZsq{}}\PY{p}{,} \PY{n}{alpha}\PY{o}{=}\PY{l+m+mf}{0.5}\PY{p}{)}

\PY{n}{x\PYZus{}pdf} \PY{o}{=} \PY{n}{np}\PY{o}{.}\PY{n}{linspace}\PY{p}{(}\PY{o}{\PYZhy{}}\PY{l+m+mf}{0.5}\PY{p}{,} \PY{l+m+mf}{1.5}\PY{p}{,} \PY{n}{train\PYZus{}dataset}\PY{o}{.}\PY{n}{pos}\PY{o}{.}\PY{n}{shape}\PY{p}{[}\PY{l+m+mi}{0}\PY{p}{]}\PY{p}{)}
\PY{n}{pos\PYZus{}pdf} \PY{o}{=} \PY{n}{pos\PYZus{}dist}\PY{o}{.}\PY{n}{pdf}\PY{p}{(}\PY{n}{x\PYZus{}pdf}\PY{p}{)}
\PY{n}{neg\PYZus{}pdf} \PY{o}{=} \PY{n}{neg\PYZus{}dist}\PY{o}{.}\PY{n}{pdf}\PY{p}{(}\PY{n}{x\PYZus{}pdf}\PY{p}{)}
\PY{n}{plt}\PY{o}{.}\PY{n}{plot}\PY{p}{(}\PY{n}{x\PYZus{}pdf}\PY{p}{,} \PY{n}{pos\PYZus{}pdf}\PY{p}{,} \PY{l+s+s1}{\PYZsq{}}\PY{l+s+s1}{blue}\PY{l+s+s1}{\PYZsq{}}\PY{p}{,} \PY{n}{linewidth}\PY{o}{=}\PY{l+m+mi}{3}\PY{p}{,} \PY{n}{label}\PY{o}{=}\PY{l+s+s2}{\PYZdq{}}\PY{l+s+s2}{pos\PYZus{}pdf}\PY{l+s+s2}{\PYZdq{}}\PY{p}{)}
\PY{n}{plt}\PY{o}{.}\PY{n}{plot}\PY{p}{(}\PY{n}{x\PYZus{}pdf}\PY{p}{,} \PY{n}{neg\PYZus{}pdf}\PY{p}{,} \PY{l+s+s1}{\PYZsq{}}\PY{l+s+s1}{red}\PY{l+s+s1}{\PYZsq{}}\PY{p}{,} \PY{n}{linewidth}\PY{o}{=}\PY{l+m+mi}{3}\PY{p}{,} \PY{n}{label}\PY{o}{=}\PY{l+s+s2}{\PYZdq{}}\PY{l+s+s2}{neg\PYZus{}pdf}\PY{l+s+s2}{\PYZdq{}}\PY{p}{)}

\PY{n}{plt}\PY{o}{.}\PY{n}{legend}\PY{p}{(}\PY{p}{)}
\PY{n}{plt}\PY{o}{.}\PY{n}{show}\PY{p}{(}\PY{p}{)}
\end{Verbatim}
\end{tcolorbox}

    \begin{center}
    \adjustimage{max size={0.9\linewidth}{0.9\paperheight}}{zadani_files/zadani_13_0.png}
    \end{center}
    { \hspace*{\fill} \\}
    
    Naimplementujte binární generativní klasifikátor. Při konstrukci přijímá
dvě rozložení poskytující metodu \texttt{.pdf()} a odpovídající apriorní
pravděpodobnost tříd. Dbejte, aby Vám uživatel nemohl zadat neplatné
apriorní pravděpodobnosti. Jako všechny klasifikátory v tomto projektu
poskytuje metodu \texttt{prob\_class\_1()}.

\textbf{(9 řádků)}

    \begin{tcolorbox}[breakable, size=fbox, boxrule=1pt, pad at break*=1mm,colback=cellbackground, colframe=cellborder]
\prompt{In}{incolor}{8}{\boxspacing}
\begin{Verbatim}[commandchars=\\\{\}]
\PY{k}{class} \PY{n+nc}{GenerativeClassifier2Class}\PY{p}{:}
    \PY{k}{def} \PY{n+nf+fm}{\PYZus{}\PYZus{}init\PYZus{}\PYZus{}}\PY{p}{(}\PY{n+nb+bp}{self}\PY{p}{,} \PY{n}{dist1}\PY{p}{,} \PY{n}{dist2}\PY{p}{,} \PY{n}{prior\PYZus{}val}\PY{p}{)}\PY{p}{:}
        \PY{n+nb+bp}{self}\PY{o}{.}\PY{n}{dist1} \PY{o}{=} \PY{n}{dist1}
        \PY{n+nb+bp}{self}\PY{o}{.}\PY{n}{dist2} \PY{o}{=} \PY{n}{dist2}
        \PY{n+nb+bp}{self}\PY{o}{.}\PY{n}{prior\PYZus{}val} \PY{o}{=} \PY{n}{prior\PYZus{}val}
        \PY{k}{if} \PY{n}{prior\PYZus{}val}\PY{o}{.}\PY{n}{any}\PY{p}{(}\PY{p}{)} \PY{o}{\PYZgt{}} \PY{l+m+mi}{1} \PY{o+ow}{or} \PY{n}{prior\PYZus{}val}\PY{o}{.}\PY{n}{any}\PY{p}{(}\PY{p}{)} \PY{o}{\PYZlt{}} \PY{l+m+mi}{0}\PY{p}{:}
            \PY{k}{raise} \PY{n+ne}{ValueError}\PY{p}{(}\PY{l+s+s2}{\PYZdq{}}\PY{l+s+s2}{Incorrect aprior...}\PY{l+s+s2}{\PYZdq{}}\PY{p}{)}
    
    \PY{k}{def} \PY{n+nf}{prob\PYZus{}class\PYZus{}1}\PY{p}{(}\PY{n+nb+bp}{self}\PY{p}{,} \PY{n}{xs}\PY{p}{)}\PY{p}{:}
        \PY{k}{return} \PY{n+nb+bp}{self}\PY{o}{.}\PY{n}{prior\PYZus{}val}\PY{p}{[}\PY{l+m+mi}{1}\PY{p}{]} \PY{o}{*} \PY{n+nb+bp}{self}\PY{o}{.}\PY{n}{dist1}\PY{o}{.}\PY{n}{pdf}\PY{p}{(}\PY{n}{xs}\PY{p}{)} \PY{o}{/} \PY{p}{(}\PY{n+nb+bp}{self}\PY{o}{.}\PY{n}{prior\PYZus{}val}\PY{p}{[}\PY{l+m+mi}{1}\PY{p}{]} \PY{o}{*} \PY{n+nb+bp}{self}\PY{o}{.}\PY{n}{dist1}\PY{o}{.}\PY{n}{pdf}\PY{p}{(}\PY{n}{xs}\PY{p}{)} \PY{o}{+} \PY{n+nb+bp}{self}\PY{o}{.}\PY{n}{prior\PYZus{}val}\PY{p}{[}\PY{l+m+mi}{0}\PY{p}{]} \PY{o}{*} \PY{n+nb+bp}{self}\PY{o}{.}\PY{n}{dist2}\PY{o}{.}\PY{n}{pdf}\PY{p}{(}\PY{n}{xs}\PY{p}{)}\PY{p}{)}
\end{Verbatim}
\end{tcolorbox}

    Nainstancujte dva generativní klasifikátory: jeden s rovnoměrnými priory
a jeden s apriorní pravděpodobností 0.75 pro třídu 0 (negativní
příklady). Pomocí funkce \texttt{evaluate()} vyhodnotíte jejich
úspěšnost na validačních datech.

\textbf{(2 řádky)}

    \begin{tcolorbox}[breakable, size=fbox, boxrule=1pt, pad at break*=1mm,colback=cellbackground, colframe=cellborder]
\prompt{In}{incolor}{9}{\boxspacing}
\begin{Verbatim}[commandchars=\\\{\}]
\PY{n}{classifier\PYZus{}flat\PYZus{}prior} \PY{o}{=} \PY{n}{GenerativeClassifier2Class}\PY{p}{(}\PY{n}{pos\PYZus{}dist}\PY{p}{,} \PY{n}{neg\PYZus{}dist}\PY{p}{,} \PY{n}{np}\PY{o}{.}\PY{n}{array}\PY{p}{(}\PY{p}{[}\PY{l+m+mf}{0.5}\PY{p}{,} \PY{l+m+mf}{0.5}\PY{p}{]}\PY{p}{)}\PY{p}{)}
\PY{n}{classifier\PYZus{}full\PYZus{}prior} \PY{o}{=} \PY{n}{GenerativeClassifier2Class}\PY{p}{(}\PY{n}{pos\PYZus{}dist}\PY{p}{,} \PY{n}{neg\PYZus{}dist}\PY{p}{,} \PY{n}{np}\PY{o}{.}\PY{n}{array}\PY{p}{(}\PY{p}{[}\PY{l+m+mf}{0.75}\PY{p}{,} \PY{l+m+mf}{0.25}\PY{p}{]}\PY{p}{)}\PY{p}{)}

\PY{n+nb}{print}\PY{p}{(}\PY{l+s+s1}{\PYZsq{}}\PY{l+s+s1}{flat:}\PY{l+s+s1}{\PYZsq{}}\PY{p}{,} \PY{n}{evaluate}\PY{p}{(}\PY{n}{classifier\PYZus{}flat\PYZus{}prior}\PY{p}{,} \PY{n}{val\PYZus{}dataset}\PY{o}{.}\PY{n}{xs}\PY{p}{[}\PY{p}{:}\PY{p}{,} \PY{n}{FOI}\PY{p}{]}\PY{p}{,} \PY{n}{val\PYZus{}dataset}\PY{o}{.}\PY{n}{targets}\PY{p}{)}\PY{p}{)}
\PY{n+nb}{print}\PY{p}{(}\PY{l+s+s1}{\PYZsq{}}\PY{l+s+s1}{full:}\PY{l+s+s1}{\PYZsq{}}\PY{p}{,} \PY{n}{evaluate}\PY{p}{(}\PY{n}{classifier\PYZus{}full\PYZus{}prior}\PY{p}{,} \PY{n}{val\PYZus{}dataset}\PY{o}{.}\PY{n}{xs}\PY{p}{[}\PY{p}{:}\PY{p}{,} \PY{n}{FOI}\PY{p}{]}\PY{p}{,} \PY{n}{val\PYZus{}dataset}\PY{o}{.}\PY{n}{targets}\PY{p}{)}\PY{p}{)}
\end{Verbatim}
\end{tcolorbox}

    \begin{Verbatim}[commandchars=\\\{\}]
flat: 0.809
full: 0.8475
    \end{Verbatim}

    Vykreslete průběh posteriorní pravděpodobnosti třídy 1 jako funkci
příznaku 5, opět v rozsahu \textless-0.5; 1.5\textgreater{} pro oba
klasifikátory. Do grafu zakreslete i histogramy rozložení trénovacích
dat, opět s \texttt{density=True} pro zachování dynamického rozsahu.

\textbf{(8 řádků)}

    \begin{tcolorbox}[breakable, size=fbox, boxrule=1pt, pad at break*=1mm,colback=cellbackground, colframe=cellborder]
\prompt{In}{incolor}{10}{\boxspacing}
\begin{Verbatim}[commandchars=\\\{\}]
\PY{n}{pos\PYZus{}samples} \PY{o}{=} \PY{n}{train\PYZus{}dataset}\PY{o}{.}\PY{n}{pos}\PY{p}{[}\PY{p}{:}\PY{p}{,} \PY{n}{FOI}\PY{p}{]}
\PY{n}{neg\PYZus{}samples} \PY{o}{=} \PY{n}{train\PYZus{}dataset}\PY{o}{.}\PY{n}{neg}\PY{p}{[}\PY{p}{:}\PY{p}{,} \PY{n}{FOI}\PY{p}{]}
\PY{n}{plt}\PY{o}{.}\PY{n}{hist}\PY{p}{(}\PY{n}{pos\PYZus{}samples}\PY{p}{,} \PY{n}{density}\PY{o}{=}\PY{k+kc}{True}\PY{p}{,} \PY{n}{bins}\PY{o}{=}\PY{l+m+mi}{20}\PY{p}{,} \PY{n}{color}\PY{o}{=}\PY{l+s+s1}{\PYZsq{}}\PY{l+s+s1}{deepskyblue}\PY{l+s+s1}{\PYZsq{}}\PY{p}{,} \PY{n}{label}\PY{o}{=}\PY{l+s+s1}{\PYZsq{}}\PY{l+s+s1}{pos}\PY{l+s+s1}{\PYZsq{}}\PY{p}{,} \PY{n}{alpha}\PY{o}{=}\PY{l+m+mf}{0.5}\PY{p}{)}
\PY{n}{plt}\PY{o}{.}\PY{n}{hist}\PY{p}{(}\PY{n}{neg\PYZus{}samples}\PY{p}{,} \PY{n}{density}\PY{o}{=}\PY{k+kc}{True}\PY{p}{,} \PY{n}{bins}\PY{o}{=}\PY{l+m+mi}{20}\PY{p}{,} \PY{n}{color}\PY{o}{=}\PY{l+s+s1}{\PYZsq{}}\PY{l+s+s1}{magenta}\PY{l+s+s1}{\PYZsq{}}\PY{p}{,} \PY{n}{label}\PY{o}{=}\PY{l+s+s1}{\PYZsq{}}\PY{l+s+s1}{neg}\PY{l+s+s1}{\PYZsq{}}\PY{p}{,} \PY{n}{alpha}\PY{o}{=}\PY{l+m+mf}{0.5}\PY{p}{)}

\PY{n}{x} \PY{o}{=} \PY{n}{np}\PY{o}{.}\PY{n}{linspace}\PY{p}{(}\PY{o}{\PYZhy{}}\PY{l+m+mf}{0.5}\PY{p}{,} \PY{l+m+mf}{1.5}\PY{p}{,} \PY{n}{train\PYZus{}dataset}\PY{o}{.}\PY{n}{pos}\PY{o}{.}\PY{n}{shape}\PY{p}{[}\PY{l+m+mi}{0}\PY{p}{]}\PY{p}{)}
\PY{n}{plt}\PY{o}{.}\PY{n}{plot}\PY{p}{(}\PY{n}{x}\PY{p}{,} \PY{n}{classifier\PYZus{}flat\PYZus{}prior}\PY{o}{.}\PY{n}{prob\PYZus{}class\PYZus{}1}\PY{p}{(}\PY{n}{x}\PY{p}{)}\PY{p}{,} \PY{l+s+s1}{\PYZsq{}}\PY{l+s+s1}{blue}\PY{l+s+s1}{\PYZsq{}}\PY{p}{,} \PY{n}{linewidth}\PY{o}{=}\PY{l+m+mi}{3}\PY{p}{,} \PY{n}{label}\PY{o}{=}\PY{l+s+s2}{\PYZdq{}}\PY{l+s+s2}{flat}\PY{l+s+s2}{\PYZdq{}}\PY{p}{)}
\PY{n}{plt}\PY{o}{.}\PY{n}{plot}\PY{p}{(}\PY{n}{x}\PY{p}{,} \PY{n}{classifier\PYZus{}full\PYZus{}prior}\PY{o}{.}\PY{n}{prob\PYZus{}class\PYZus{}1}\PY{p}{(}\PY{n}{x}\PY{p}{)}\PY{p}{,} \PY{l+s+s1}{\PYZsq{}}\PY{l+s+s1}{red}\PY{l+s+s1}{\PYZsq{}}\PY{p}{,} \PY{n}{linewidth}\PY{o}{=}\PY{l+m+mi}{3}\PY{p}{,} \PY{n}{label}\PY{o}{=}\PY{l+s+s2}{\PYZdq{}}\PY{l+s+s2}{full}\PY{l+s+s2}{\PYZdq{}}\PY{p}{)}

\PY{n}{plt}\PY{o}{.}\PY{n}{legend}\PY{p}{(}\PY{p}{)}
\PY{n}{plt}\PY{o}{.}\PY{n}{show}\PY{p}{(}\PY{p}{)}
\end{Verbatim}
\end{tcolorbox}

    \begin{center}
    \adjustimage{max size={0.9\linewidth}{0.9\paperheight}}{zadani_files/zadani_19_0.png}
    \end{center}
    { \hspace*{\fill} \\}
    
    \section{Diskriminativní
klasifikátory}\label{diskriminativnuxed-klasifikuxe1tory}

V následující části budete pomocí (lineární) logistické regrese přímo
modelovat posteriorní pravděpodobnost třídy 1. Modely budou založeny
čistě na NumPy, takže nemusíte instalovat nic dalšího. Nabitějších
toolkitů se dočkáte ve třetím projektu.

    \begin{tcolorbox}[breakable, size=fbox, boxrule=1pt, pad at break*=1mm,colback=cellbackground, colframe=cellborder]
\prompt{In}{incolor}{11}{\boxspacing}
\begin{Verbatim}[commandchars=\\\{\}]
\PY{k}{def} \PY{n+nf}{logistic\PYZus{}sigmoid}\PY{p}{(}\PY{n}{x}\PY{p}{)}\PY{p}{:}
    \PY{k}{return} \PY{n}{np}\PY{o}{.}\PY{n}{exp}\PY{p}{(}\PY{o}{\PYZhy{}}\PY{n}{np}\PY{o}{.}\PY{n}{logaddexp}\PY{p}{(}\PY{l+m+mi}{0}\PY{p}{,} \PY{o}{\PYZhy{}}\PY{n}{x}\PY{p}{)}\PY{p}{)}

\PY{k}{def} \PY{n+nf}{binary\PYZus{}cross\PYZus{}entropy}\PY{p}{(}\PY{n}{probs}\PY{p}{,} \PY{n}{targets}\PY{p}{)}\PY{p}{:}
    \PY{k}{return} \PY{n}{np}\PY{o}{.}\PY{n}{sum}\PY{p}{(}\PY{o}{\PYZhy{}}\PY{n}{targets} \PY{o}{*} \PY{n}{np}\PY{o}{.}\PY{n}{log}\PY{p}{(}\PY{n}{probs}\PY{p}{)} \PY{o}{\PYZhy{}} \PY{p}{(}\PY{l+m+mi}{1}\PY{o}{\PYZhy{}}\PY{n}{targets}\PY{p}{)}\PY{o}{*}\PY{n}{np}\PY{o}{.}\PY{n}{log}\PY{p}{(}\PY{l+m+mi}{1}\PY{o}{\PYZhy{}}\PY{n}{probs}\PY{p}{)}\PY{p}{)} 

\PY{k}{class} \PY{n+nc}{LogisticRegressionNumpy}\PY{p}{:}
    \PY{k}{def} \PY{n+nf+fm}{\PYZus{}\PYZus{}init\PYZus{}\PYZus{}}\PY{p}{(}\PY{n+nb+bp}{self}\PY{p}{,} \PY{n}{dim}\PY{p}{)}\PY{p}{:}
        \PY{n+nb+bp}{self}\PY{o}{.}\PY{n}{w} \PY{o}{=} \PY{n}{np}\PY{o}{.}\PY{n}{array}\PY{p}{(}\PY{p}{[}\PY{l+m+mf}{0.0}\PY{p}{]} \PY{o}{*} \PY{n}{dim}\PY{p}{)}
        \PY{n+nb+bp}{self}\PY{o}{.}\PY{n}{b} \PY{o}{=} \PY{n}{np}\PY{o}{.}\PY{n}{array}\PY{p}{(}\PY{p}{[}\PY{l+m+mf}{0.0}\PY{p}{]}\PY{p}{)}
        
    \PY{k}{def} \PY{n+nf}{prob\PYZus{}class\PYZus{}1}\PY{p}{(}\PY{n+nb+bp}{self}\PY{p}{,} \PY{n}{x}\PY{p}{)}\PY{p}{:}
        \PY{k}{return} \PY{n}{logistic\PYZus{}sigmoid}\PY{p}{(}\PY{n}{x} \PY{o}{@} \PY{n+nb+bp}{self}\PY{o}{.}\PY{n}{w} \PY{o}{+} \PY{n+nb+bp}{self}\PY{o}{.}\PY{n}{b}\PY{p}{)}
\end{Verbatim}
\end{tcolorbox}

    Diskriminativní klasifikátor očekává, že dostane vstup ve tvaru
\texttt{{[}N,\ F{]}}. Pro práci na jediném příznaku bude tedy zapotřebí
vyřezávat příslušná data v správném formátu (\texttt{{[}N,\ 1{]}}).
Doimplementujte třídu \texttt{FeatureCutter} tak, aby to zařizovalo
volání její instance. Který příznak se použije, nechť je konfigurováno
při konstrukci.

Může se Vám hodit \texttt{np.newaxis}.

\textbf{(2 řádky)}

    \begin{tcolorbox}[breakable, size=fbox, boxrule=1pt, pad at break*=1mm,colback=cellbackground, colframe=cellborder]
\prompt{In}{incolor}{12}{\boxspacing}
\begin{Verbatim}[commandchars=\\\{\}]
\PY{k}{class} \PY{n+nc}{FeatureCutter}\PY{p}{:}
    \PY{k}{def} \PY{n+nf+fm}{\PYZus{}\PYZus{}init\PYZus{}\PYZus{}}\PY{p}{(}\PY{n+nb+bp}{self}\PY{p}{,} \PY{n}{fea\PYZus{}id}\PY{p}{)}\PY{p}{:}
        \PY{n+nb+bp}{self}\PY{o}{.}\PY{n}{fea\PYZus{}id} \PY{o}{=} \PY{n}{fea\PYZus{}id}
        
    \PY{k}{def} \PY{n+nf+fm}{\PYZus{}\PYZus{}call\PYZus{}\PYZus{}}\PY{p}{(}\PY{n+nb+bp}{self}\PY{p}{,} \PY{n}{x} \PY{p}{:} \PY{n}{np}\PY{o}{.}\PY{n}{ndarray}\PY{p}{)} \PY{o}{\PYZhy{}}\PY{o}{\PYZgt{}} \PY{n}{np}\PY{o}{.}\PY{n}{ndarray}\PY{p}{:}
        \PY{k}{return} \PY{n}{x}\PY{p}{[}\PY{p}{:}\PY{p}{,} \PY{n+nb+bp}{self}\PY{o}{.}\PY{n}{fea\PYZus{}id}\PY{p}{,} \PY{n}{np}\PY{o}{.}\PY{n}{newaxis}\PY{p}{]}
\end{Verbatim}
\end{tcolorbox}

    Dalším krokem je implementovat funkci, která model vytvoří a natrénuje.
Jejím výstupem bude (1) natrénovaný model, (2) průběh trénovací loss a
(3) průběh validační přesnosti. Neuvažujte žádné minibatche,
aktualizujte váhy vždy na celém trénovacím datasetu. Po každém kroku
vyhodnoťte model na validačních datech. Jako model vracejte ten, který
dosáhne nejlepší validační přesnosti. Jako loss použijte binární
cross-entropii a logujte průměr na vzorek. Pro výpočet validační
přesnosti využijte funkci \texttt{evaluate()}. Oba průběhy vracejte jako
obyčejné seznamy.

\textbf{(cca 11 řádků)}

    \begin{tcolorbox}[breakable, size=fbox, boxrule=1pt, pad at break*=1mm,colback=cellbackground, colframe=cellborder]
\prompt{In}{incolor}{13}{\boxspacing}
\begin{Verbatim}[commandchars=\\\{\}]
\PY{k}{def} \PY{n+nf}{train\PYZus{}logistic\PYZus{}regression}\PY{p}{(}\PY{n}{nb\PYZus{}epochs}\PY{p}{,} \PY{n}{lr}\PY{p}{,} \PY{n}{in\PYZus{}dim}\PY{p}{,} \PY{n}{fea\PYZus{}preprocessor}\PY{p}{)}\PY{p}{:}
    \PY{n}{model} \PY{o}{=} \PY{n}{LogisticRegressionNumpy}\PY{p}{(}\PY{n}{in\PYZus{}dim}\PY{p}{)}
    \PY{n}{best\PYZus{}model} \PY{p}{:} \PY{n}{LogisticRegressionNumpy} \PY{o}{=} \PY{n}{copy}\PY{o}{.}\PY{n}{deepcopy}\PY{p}{(}\PY{n}{model}\PY{p}{)}
    \PY{n}{losses} \PY{o}{=} \PY{p}{[}\PY{p}{]}
    \PY{n}{accuracies} \PY{o}{=} \PY{p}{[}\PY{p}{]}
    
    \PY{n}{train\PYZus{}X} \PY{o}{=} \PY{n}{fea\PYZus{}preprocessor}\PY{p}{(}\PY{n}{train\PYZus{}dataset}\PY{o}{.}\PY{n}{xs}\PY{p}{)}
    \PY{n}{train\PYZus{}t} \PY{o}{=} \PY{n}{train\PYZus{}dataset}\PY{o}{.}\PY{n}{targets}

    \PY{c+c1}{\PYZsh{} validation DS}
    \PY{n}{val\PYZus{}X} \PY{o}{=} \PY{n}{fea\PYZus{}preprocessor}\PY{p}{(}\PY{n}{val\PYZus{}dataset}\PY{o}{.}\PY{n}{xs}\PY{p}{)}
    \PY{n}{val\PYZus{}t} \PY{o}{=} \PY{n}{val\PYZus{}dataset}\PY{o}{.}\PY{n}{targets}

    \PY{n}{n} \PY{o}{=} \PY{n}{train\PYZus{}X}\PY{o}{.}\PY{n}{shape}\PY{p}{[}\PY{l+m+mi}{0}\PY{p}{]}

    \PY{n}{max\PYZus{}accuracy} \PY{o}{=} \PY{l+m+mf}{0.0}

    \PY{k}{for} \PY{n}{epoch} \PY{o+ow}{in} \PY{n+nb}{range}\PY{p}{(}\PY{n}{nb\PYZus{}epochs}\PY{p}{)}\PY{p}{:}

        \PY{n}{probs} \PY{o}{=} \PY{n}{model}\PY{o}{.}\PY{n}{prob\PYZus{}class\PYZus{}1}\PY{p}{(}\PY{n}{train\PYZus{}X}\PY{p}{)}

        \PY{n}{dw} \PY{o}{=} \PY{n}{np}\PY{o}{.}\PY{n}{dot}\PY{p}{(}\PY{n}{train\PYZus{}X}\PY{o}{.}\PY{n}{T}\PY{p}{,} \PY{p}{(}\PY{n}{probs} \PY{o}{\PYZhy{}} \PY{n}{train\PYZus{}t}\PY{p}{)}\PY{p}{)}
        \PY{n}{db} \PY{o}{=} \PY{n}{np}\PY{o}{.}\PY{n}{sum}\PY{p}{(}\PY{n}{probs} \PY{o}{\PYZhy{}} \PY{n}{train\PYZus{}t}\PY{p}{)}

        \PY{n}{model}\PY{o}{.}\PY{n}{w} \PY{o}{\PYZhy{}}\PY{o}{=} \PY{n}{lr} \PY{o}{*} \PY{n}{dw}
        \PY{n}{model}\PY{o}{.}\PY{n}{b} \PY{o}{\PYZhy{}}\PY{o}{=} \PY{n}{lr} \PY{o}{*} \PY{n}{db}

        \PY{n}{losses}\PY{o}{.}\PY{n}{append}\PY{p}{(}\PY{n}{binary\PYZus{}cross\PYZus{}entropy}\PY{p}{(}\PY{n}{probs}\PY{p}{,} \PY{n}{train\PYZus{}t}\PY{p}{)} \PY{o}{/} \PY{n}{n}\PY{p}{)}

        \PY{n}{accuracy} \PY{o}{=} \PY{n}{evaluate}\PY{p}{(}\PY{n}{model}\PY{p}{,} \PY{n}{val\PYZus{}X}\PY{p}{,} \PY{n}{val\PYZus{}t}\PY{p}{)}
        \PY{n}{accuracies}\PY{o}{.}\PY{n}{append}\PY{p}{(}\PY{n}{accuracy}\PY{p}{)}
        
        \PY{k}{if} \PY{n}{accuracy} \PY{o}{\PYZgt{}} \PY{n}{max\PYZus{}accuracy}\PY{p}{:}
            \PY{n}{max\PYZus{}accuracy} \PY{o}{=} \PY{n}{accuracy}
            \PY{n}{best\PYZus{}model} \PY{o}{=} \PY{n}{copy}\PY{o}{.}\PY{n}{deepcopy}\PY{p}{(}\PY{n}{model}\PY{p}{)}

        
    \PY{k}{return} \PY{n}{best\PYZus{}model}\PY{p}{,} \PY{n}{losses}\PY{p}{,} \PY{n}{accuracies}
\end{Verbatim}
\end{tcolorbox}

    Funkci zavolejte a natrénujte model. Uveďte zde parametry, které vám
dají slušný výsledek. Měli byste dostat přesnost srovnatelnou s
generativním klasifikátorem s nastavenými priory. Neměli byste
potřebovat víc, než 100 epoch. Vykreslete průběh trénovací loss a
validační přesnosti, osu x značte v epochách.

V druhém grafu vykreslete histogramy trénovacích dat a pravděpodobnost
třídy 1 pro x od -0.5 do 1.5, podobně jako výše u generativních
klasifikátorů.

\textbf{(1 + 5 + 8 řádků)}

    \begin{tcolorbox}[breakable, size=fbox, boxrule=1pt, pad at break*=1mm,colback=cellbackground, colframe=cellborder]
\prompt{In}{incolor}{14}{\boxspacing}
\begin{Verbatim}[commandchars=\\\{\}]
\PY{n}{disc\PYZus{}fea5}\PY{p}{,} \PY{n}{losses}\PY{p}{,} \PY{n}{accuracies} \PY{o}{=} \PY{n}{train\PYZus{}logistic\PYZus{}regression}\PY{p}{(}\PY{n}{nb\PYZus{}epochs}\PY{o}{=}\PY{l+m+mi}{100}\PY{p}{,} \PY{n}{lr}\PY{o}{=}\PY{l+m+mf}{0.0003}\PY{p}{,} \PY{n}{in\PYZus{}dim}\PY{o}{=}\PY{l+m+mi}{1}\PY{p}{,} \PY{n}{fea\PYZus{}preprocessor}\PY{o}{=} \PY{n}{FeatureCutter}\PY{p}{(}\PY{n}{FOI}\PY{p}{)}\PY{p}{)}

\PY{n}{plt}\PY{o}{.}\PY{n}{plot}\PY{p}{(}\PY{n+nb}{range}\PY{p}{(}\PY{l+m+mi}{1}\PY{p}{,} \PY{n+nb}{len}\PY{p}{(}\PY{n}{losses}\PY{p}{)} \PY{o}{+} \PY{l+m+mi}{1}\PY{p}{)}\PY{p}{,} \PY{n}{losses}\PY{p}{,} \PY{n}{label}\PY{o}{=}\PY{l+s+s1}{\PYZsq{}}\PY{l+s+s1}{Training Loss}\PY{l+s+s1}{\PYZsq{}}\PY{p}{,} \PY{n}{color}\PY{o}{=}\PY{l+s+s1}{\PYZsq{}}\PY{l+s+s1}{blue}\PY{l+s+s1}{\PYZsq{}}\PY{p}{)}
\PY{n}{plt}\PY{o}{.}\PY{n}{plot}\PY{p}{(}\PY{n+nb}{range}\PY{p}{(}\PY{l+m+mi}{1}\PY{p}{,} \PY{n+nb}{len}\PY{p}{(}\PY{n}{accuracies}\PY{p}{)} \PY{o}{+} \PY{l+m+mi}{1}\PY{p}{)}\PY{p}{,} \PY{n}{accuracies}\PY{p}{,} \PY{n}{label}\PY{o}{=}\PY{l+s+s1}{\PYZsq{}}\PY{l+s+s1}{Validation Accuracy}\PY{l+s+s1}{\PYZsq{}}\PY{p}{,} \PY{n}{color}\PY{o}{=}\PY{l+s+s1}{\PYZsq{}}\PY{l+s+s1}{red}\PY{l+s+s1}{\PYZsq{}}\PY{p}{)}
\PY{n}{plt}\PY{o}{.}\PY{n}{xlabel}\PY{p}{(}\PY{l+s+s1}{\PYZsq{}}\PY{l+s+s1}{Epoch}\PY{l+s+s1}{\PYZsq{}}\PY{p}{)}
\PY{n}{plt}\PY{o}{.}\PY{n}{ylabel}\PY{p}{(}\PY{l+s+s1}{\PYZsq{}}\PY{l+s+s1}{Accuracy/Loss}\PY{l+s+s1}{\PYZsq{}}\PY{p}{)}
\PY{n}{plt}\PY{o}{.}\PY{n}{title}\PY{p}{(}\PY{l+s+s1}{\PYZsq{}}\PY{l+s+s1}{Training Loss and Validation Accuracy Over Epochs}\PY{l+s+s1}{\PYZsq{}}\PY{p}{)}
\PY{n}{plt}\PY{o}{.}\PY{n}{legend}\PY{p}{(}\PY{p}{)}
\PY{n}{plt}\PY{o}{.}\PY{n}{show}\PY{p}{(}\PY{p}{)}

\PY{n+nb}{print}\PY{p}{(}\PY{l+s+s1}{\PYZsq{}}\PY{l+s+s1}{w}\PY{l+s+s1}{\PYZsq{}}\PY{p}{,} \PY{n}{disc\PYZus{}fea5}\PY{o}{.}\PY{n}{w}\PY{o}{.}\PY{n}{item}\PY{p}{(}\PY{p}{)}\PY{p}{,} \PY{l+s+s1}{\PYZsq{}}\PY{l+s+s1}{b}\PY{l+s+s1}{\PYZsq{}}\PY{p}{,} \PY{n}{disc\PYZus{}fea5}\PY{o}{.}\PY{n}{b}\PY{o}{.}\PY{n}{item}\PY{p}{(}\PY{p}{)}\PY{p}{)}

\PY{n}{pos\PYZus{}samples} \PY{o}{=} \PY{n}{train\PYZus{}dataset}\PY{o}{.}\PY{n}{pos}\PY{p}{[}\PY{p}{:}\PY{p}{,} \PY{n}{FOI}\PY{p}{]}
\PY{n}{neg\PYZus{}samples} \PY{o}{=} \PY{n}{train\PYZus{}dataset}\PY{o}{.}\PY{n}{neg}\PY{p}{[}\PY{p}{:}\PY{p}{,} \PY{n}{FOI}\PY{p}{]}
\PY{n}{plt}\PY{o}{.}\PY{n}{hist}\PY{p}{(}\PY{n}{pos\PYZus{}samples}\PY{p}{,} \PY{n}{density}\PY{o}{=}\PY{k+kc}{True}\PY{p}{,} \PY{n}{bins}\PY{o}{=}\PY{l+m+mi}{20}\PY{p}{,} \PY{n}{color}\PY{o}{=}\PY{l+s+s1}{\PYZsq{}}\PY{l+s+s1}{deepskyblue}\PY{l+s+s1}{\PYZsq{}}\PY{p}{,} \PY{n}{label}\PY{o}{=}\PY{l+s+s1}{\PYZsq{}}\PY{l+s+s1}{pos}\PY{l+s+s1}{\PYZsq{}}\PY{p}{,} \PY{n}{alpha}\PY{o}{=}\PY{l+m+mf}{0.5}\PY{p}{)}
\PY{n}{plt}\PY{o}{.}\PY{n}{hist}\PY{p}{(}\PY{n}{neg\PYZus{}samples}\PY{p}{,} \PY{n}{density}\PY{o}{=}\PY{k+kc}{True}\PY{p}{,} \PY{n}{bins}\PY{o}{=}\PY{l+m+mi}{20}\PY{p}{,} \PY{n}{color}\PY{o}{=}\PY{l+s+s1}{\PYZsq{}}\PY{l+s+s1}{magenta}\PY{l+s+s1}{\PYZsq{}}\PY{p}{,} \PY{n}{label}\PY{o}{=}\PY{l+s+s1}{\PYZsq{}}\PY{l+s+s1}{neg}\PY{l+s+s1}{\PYZsq{}}\PY{p}{,} \PY{n}{alpha}\PY{o}{=}\PY{l+m+mf}{0.5}\PY{p}{)}
\PY{n}{x} \PY{o}{=} \PY{n}{np}\PY{o}{.}\PY{n}{linspace}\PY{p}{(}\PY{o}{\PYZhy{}}\PY{l+m+mf}{0.5}\PY{p}{,} \PY{l+m+mf}{1.5}\PY{p}{,} \PY{n}{train\PYZus{}dataset}\PY{o}{.}\PY{n}{pos}\PY{o}{.}\PY{n}{shape}\PY{p}{[}\PY{l+m+mi}{0}\PY{p}{]}\PY{p}{)}
\PY{n}{plt}\PY{o}{.}\PY{n}{plot}\PY{p}{(}\PY{n}{x}\PY{p}{,} \PY{n}{disc\PYZus{}fea5}\PY{o}{.}\PY{n}{prob\PYZus{}class\PYZus{}1}\PY{p}{(}\PY{n}{x}\PY{p}{[}\PY{p}{:}\PY{p}{,} \PY{n}{np}\PY{o}{.}\PY{n}{newaxis}\PY{p}{]}\PY{p}{)}\PY{p}{,} \PY{l+s+s1}{\PYZsq{}}\PY{l+s+s1}{red}\PY{l+s+s1}{\PYZsq{}}\PY{p}{,} \PY{n}{linewidth}\PY{o}{=}\PY{l+m+mi}{3}\PY{p}{,} \PY{n}{label}\PY{o}{=}\PY{l+s+s2}{\PYZdq{}}\PY{l+s+s2}{disc\PYZus{}fea5}\PY{l+s+s2}{\PYZdq{}}\PY{p}{)}
\PY{n}{plt}\PY{o}{.}\PY{n}{legend}\PY{p}{(}\PY{p}{)}
\PY{n}{plt}\PY{o}{.}\PY{n}{show}\PY{p}{(}\PY{p}{)}

\PY{n+nb}{print}\PY{p}{(}\PY{l+s+s1}{\PYZsq{}}\PY{l+s+s1}{disc\PYZus{}fea5:}\PY{l+s+s1}{\PYZsq{}}\PY{p}{,} \PY{n}{evaluate}\PY{p}{(}\PY{n}{disc\PYZus{}fea5}\PY{p}{,} \PY{n}{val\PYZus{}dataset}\PY{o}{.}\PY{n}{xs}\PY{p}{[}\PY{p}{:}\PY{p}{,} \PY{n}{FOI}\PY{p}{]}\PY{p}{[}\PY{p}{:}\PY{p}{,} \PY{n}{np}\PY{o}{.}\PY{n}{newaxis}\PY{p}{]}\PY{p}{,} \PY{n}{val\PYZus{}dataset}\PY{o}{.}\PY{n}{targets}\PY{p}{)}\PY{p}{)}
\end{Verbatim}
\end{tcolorbox}

    \begin{center}
    \adjustimage{max size={0.9\linewidth}{0.9\paperheight}}{zadani_files/zadani_27_0.png}
    \end{center}
    { \hspace*{\fill} \\}
    
    \begin{Verbatim}[commandchars=\\\{\}]
w 5.522679262749397 b -2.7607703207028123
    \end{Verbatim}

    \begin{center}
    \adjustimage{max size={0.9\linewidth}{0.9\paperheight}}{zadani_files/zadani_27_2.png}
    \end{center}
    { \hspace*{\fill} \\}
    
    \begin{Verbatim}[commandchars=\\\{\}]
disc\_fea5: 0.84
    \end{Verbatim}

    \subsection{Všechny vstupní
příznaky}\label{vux161echny-vstupnuxed-pux159uxedznaky}

V posledním cvičení natrénujete logistickou regresi, která využije
všechn sedm vstupních příznaků. Zavolejte funkci z předchozího cvičení,
opět vykreslete průběh trénovací loss a validační přesnosti. Měli byste
se dostat nad 90 \% přesnosti.

Může se Vám hodit \texttt{lambda} funkce.

\textbf{(1 + 5 řádků)}

    \begin{tcolorbox}[breakable, size=fbox, boxrule=1pt, pad at break*=1mm,colback=cellbackground, colframe=cellborder]
\prompt{In}{incolor}{15}{\boxspacing}
\begin{Verbatim}[commandchars=\\\{\}]
\PY{n}{disc\PYZus{}full\PYZus{}fea}\PY{p}{,} \PY{n}{losses}\PY{p}{,} \PY{n}{accuracies} \PY{o}{=} \PY{n}{train\PYZus{}logistic\PYZus{}regression}\PY{p}{(}\PY{n}{nb\PYZus{}epochs}\PY{o}{=}\PY{l+m+mi}{250}\PY{p}{,} \PY{n}{lr}\PY{o}{=}\PY{l+m+mf}{0.0000004}\PY{p}{,} \PY{n}{in\PYZus{}dim}\PY{o}{=}\PY{l+m+mi}{7}\PY{p}{,} \PY{n}{fea\PYZus{}preprocessor}\PY{o}{=}\PY{k}{lambda} \PY{n}{x}\PY{p}{:} \PY{n}{x}\PY{p}{)}

\PY{n+nb}{print}\PY{p}{(}\PY{l+s+s2}{\PYZdq{}}\PY{l+s+s2}{Maximum validation accuracy: }\PY{l+s+s2}{\PYZdq{}} \PY{o}{+} \PY{n+nb}{str}\PY{p}{(}\PY{n+nb}{max}\PY{p}{(}\PY{n}{accuracies}\PY{p}{)}\PY{p}{)}\PY{p}{)}

\PY{n}{plt}\PY{o}{.}\PY{n}{plot}\PY{p}{(}\PY{n+nb}{range}\PY{p}{(}\PY{l+m+mi}{1}\PY{p}{,} \PY{n+nb}{len}\PY{p}{(}\PY{n}{losses}\PY{p}{)} \PY{o}{+} \PY{l+m+mi}{1}\PY{p}{)}\PY{p}{,} \PY{n}{losses}\PY{p}{,} \PY{n}{label}\PY{o}{=}\PY{l+s+s1}{\PYZsq{}}\PY{l+s+s1}{Training Loss}\PY{l+s+s1}{\PYZsq{}}\PY{p}{,} \PY{n}{color}\PY{o}{=}\PY{l+s+s1}{\PYZsq{}}\PY{l+s+s1}{blue}\PY{l+s+s1}{\PYZsq{}}\PY{p}{)}
\PY{n}{plt}\PY{o}{.}\PY{n}{plot}\PY{p}{(}\PY{n+nb}{range}\PY{p}{(}\PY{l+m+mi}{1}\PY{p}{,} \PY{n+nb}{len}\PY{p}{(}\PY{n}{accuracies}\PY{p}{)} \PY{o}{+} \PY{l+m+mi}{1}\PY{p}{)}\PY{p}{,} \PY{n}{accuracies}\PY{p}{,} \PY{n}{label}\PY{o}{=}\PY{l+s+s1}{\PYZsq{}}\PY{l+s+s1}{Validation Accuracy}\PY{l+s+s1}{\PYZsq{}}\PY{p}{,} \PY{n}{color}\PY{o}{=}\PY{l+s+s1}{\PYZsq{}}\PY{l+s+s1}{red}\PY{l+s+s1}{\PYZsq{}}\PY{p}{)}
\PY{n}{plt}\PY{o}{.}\PY{n}{xlabel}\PY{p}{(}\PY{l+s+s1}{\PYZsq{}}\PY{l+s+s1}{Epoch}\PY{l+s+s1}{\PYZsq{}}\PY{p}{)}
\PY{n}{plt}\PY{o}{.}\PY{n}{ylabel}\PY{p}{(}\PY{l+s+s1}{\PYZsq{}}\PY{l+s+s1}{Accuracy/Loss}\PY{l+s+s1}{\PYZsq{}}\PY{p}{)}
\PY{n}{plt}\PY{o}{.}\PY{n}{title}\PY{p}{(}\PY{l+s+s1}{\PYZsq{}}\PY{l+s+s1}{Training Loss and Validation Accuracy Over Epochs}\PY{l+s+s1}{\PYZsq{}}\PY{p}{)}
\PY{n}{plt}\PY{o}{.}\PY{n}{legend}\PY{p}{(}\PY{p}{)}
\PY{n}{plt}\PY{o}{.}\PY{n}{show}\PY{p}{(}\PY{p}{)}
\end{Verbatim}
\end{tcolorbox}

    \begin{Verbatim}[commandchars=\\\{\}]
Maximum validation accuracy: 0.9175
    \end{Verbatim}

    \begin{center}
    \adjustimage{max size={0.9\linewidth}{0.9\paperheight}}{zadani_files/zadani_29_1.png}
    \end{center}
    { \hspace*{\fill} \\}
    
    \section{Závěrem}\label{zuxe1vux11brem}

Konečně vyhodnoťte všech pět vytvořených klasifikátorů na testovacích
datech. Stačí doplnit jejich názvy a předat jim odpovídající příznaky.
Nezapomeňte, že u logistické regrese musíte zopakovat formátovací krok z
\texttt{FeatureCutter}u.

    \begin{tcolorbox}[breakable, size=fbox, boxrule=1pt, pad at break*=1mm,colback=cellbackground, colframe=cellborder]
\prompt{In}{incolor}{16}{\boxspacing}
\begin{Verbatim}[commandchars=\\\{\}]
\PY{n}{xs\PYZus{}full} \PY{o}{=} \PY{n}{test\PYZus{}dataset}\PY{o}{.}\PY{n}{xs}
\PY{n}{xs\PYZus{}foi} \PY{o}{=} \PY{n}{test\PYZus{}dataset}\PY{o}{.}\PY{n}{xs}\PY{p}{[}\PY{p}{:}\PY{p}{,} \PY{n}{FOI}\PY{p}{]}
\PY{n}{targets} \PY{o}{=} \PY{n}{test\PYZus{}dataset}\PY{o}{.}\PY{n}{targets}

\PY{n+nb}{print}\PY{p}{(}\PY{l+s+s1}{\PYZsq{}}\PY{l+s+s1}{Baseline:}\PY{l+s+s1}{\PYZsq{}}\PY{p}{,} \PY{n}{evaluate}\PY{p}{(}\PY{n}{baseline}\PY{p}{,} \PY{n}{xs\PYZus{}full}\PY{p}{,} \PY{n}{targets}\PY{p}{)}\PY{p}{)}
\PY{n+nb}{print}\PY{p}{(}\PY{l+s+s1}{\PYZsq{}}\PY{l+s+s1}{Generative classifier (w/o prior):}\PY{l+s+s1}{\PYZsq{}}\PY{p}{,} \PY{n}{evaluate}\PY{p}{(}\PY{n}{classifier\PYZus{}flat\PYZus{}prior}\PY{p}{,} \PY{n}{xs\PYZus{}foi}\PY{p}{,} \PY{n}{targets}\PY{p}{)}\PY{p}{)}
\PY{n+nb}{print}\PY{p}{(}\PY{l+s+s1}{\PYZsq{}}\PY{l+s+s1}{Generative classifier (correct):}\PY{l+s+s1}{\PYZsq{}}\PY{p}{,} \PY{n}{evaluate}\PY{p}{(}\PY{n}{classifier\PYZus{}full\PYZus{}prior}\PY{p}{,} \PY{n}{xs\PYZus{}foi}\PY{p}{,} \PY{n}{targets}\PY{p}{)}\PY{p}{)}
\PY{n+nb}{print}\PY{p}{(}\PY{l+s+s1}{\PYZsq{}}\PY{l+s+s1}{Logistic regression:}\PY{l+s+s1}{\PYZsq{}}\PY{p}{,} \PY{n}{evaluate}\PY{p}{(}\PY{n}{disc\PYZus{}fea5}\PY{p}{,} \PY{n}{xs\PYZus{}foi}\PY{p}{[}\PY{p}{:}\PY{p}{,} \PY{n}{np}\PY{o}{.}\PY{n}{newaxis}\PY{p}{]}\PY{p}{,} \PY{n}{targets}\PY{p}{)}\PY{p}{)}
\PY{n+nb}{print}\PY{p}{(}\PY{l+s+s1}{\PYZsq{}}\PY{l+s+s1}{logistic regression all features:}\PY{l+s+s1}{\PYZsq{}}\PY{p}{,} \PY{n}{evaluate}\PY{p}{(}\PY{n}{disc\PYZus{}full\PYZus{}fea}\PY{p}{,} \PY{n}{xs\PYZus{}full}\PY{p}{,} \PY{n}{targets}\PY{p}{)}\PY{p}{)}
\end{Verbatim}
\end{tcolorbox}

    \begin{Verbatim}[commandchars=\\\{\}]
Baseline: 0.75
Generative classifier (w/o prior): 0.8
Generative classifier (correct): 0.847
Logistic regression: 0.85
logistic regression all features: 0.914
    \end{Verbatim}

    Blahopřejeme ke zvládnutí projektu! Nezapomeňte spustit celý notebook
načisto (Kernel -\textgreater{} Restart \& Run all) a zkontrolovat, že
všechny výpočty prošly podle očekávání.

Mimochodem, vstupní data nejsou synteticky generovaná. Nasbírali jsme je
z baseline řešení historicky prvního SUI projektu; vaše klasifikátory v
tomto projektu predikují, že daný hráč vyhraje dicewars, takže by se
daly použít jako heuristika pro ohodnocování listových uzlů ve stavovém
prostoru hry. Pro představu, data jsou z pozic pět kol před koncem
partie pro daného hráče. Poskytnuté příznaky popisují globální
charakteristiky stavu hry jako je například poměr délky hranic
předmětného hráče k ostatním hranicím. Nejeden projekt v ročníku 2020
realizoval požadované ``strojové učení'' kopií domácí úlohy.

    \begin{tcolorbox}[breakable, size=fbox, boxrule=1pt, pad at break*=1mm,colback=cellbackground, colframe=cellborder]
\prompt{In}{incolor}{ }{\boxspacing}
\begin{Verbatim}[commandchars=\\\{\}]

\end{Verbatim}
\end{tcolorbox}


    % Add a bibliography block to the postdoc
    
    
    
\end{document}
